\chapter{Conclusion}
\label{Conclusion}
\graphicspath{{Figures/Conclusion/}{Figures/Common/}}

Is it possible to produce a continuous coherent atomic source with sufficient flux to be viable for use in atomic interferometry?  That is the question that this thesis has sought to contribute to.  For the specific realisations of the pumping mechanisms considered in this thesis, the answer is no; however the answer to the question more generally is less clear.  

For the optical pumping mechanism considered in \chapterref{OpticalPumping}, the achievable flux of the atom laser is limited by the rate at which condensates can be produced to replace the source condensate as it is depleted.  Even if all the atoms were transferred to the lasing condensate with perfect efficiency, the continuous atom laser produced could never exceed the average flux with which the source condensates can be produced.  The optical pumping experiment described in that chapter is therefore not a useful source for atom interferometry in its present form.  However, there is some evidence that the harmful reabsorption processes in that experiment may be suppressed due to a quantum-mechanical interference effect.  A better theoretical understanding of this process would determine whether or not the source condensate could be replaced with a thermal source.  If possible, this would greatly increase the potential flux of the produced atom laser.

The evaporatively-driven pumping mechanism considered in \chapterref{KineticTheory}, is supplied by a thermal source and therefore it could be possible for it to provide a condensed source of atoms at a similar flux to the thermal source.  However, the actual flux is necessarily lower as the evaporation process removes atoms from the system as part of the pumping mechanism.  One of the aims of \chapterref{KineticTheory} was to determine just how much lower the flux of the condensed source would be.  It was found that the largest flux achievable from an available thermal atomic source was $\sim\unit[10^5]{s\textsuperscript{-1}}$, an average flux that is also achievable by producing independent condensates.

While neither of the proposed pumped atom lasers would be competitive with thermal sources for atom interferometry, pumping could improve other atom laser experiments.  In particular, experiments in which squeezing or entanglement of atomic beams is being produced and measured.  In these experiments, the atomic beam will be condensed as it would be more difficult to detect squeezing or entanglement with a thermal source.  For these experiments, a pumped atom laser could potentially offer a narrower linewidth, increased flux, and continuous operation, all of which would increase the sensitivity of measurements of atomic squeezing and entanglement.

\chapterref{Peaks} discussed one such process that leads to the formation of entangled atomic beams.  Although in the experiment discussed in that chapter it would only be possible to measure number-difference squeezing and correlations in atom detection events, a different experiment was proposed in which it may be possible to measure the entanglement of the produced atomic beams.  This chapter has also demonstrated the excellent agreement that is achievable between theory and experiment for BEC systems.  This is possible due to extraordinary degree of control over external noise sources that is possible in experiments, and the effectiveness of the theoretical techniques available for modelling them.




Now for a strong finish.


% Contrary to what has been expected previously, this chapter has demonstrated that optical pumping of an atom laser is feasible without making the condensate so narrow as to make it essentially transparent in at least one dimension.  The next step in the investigation of this pumping process must be the theoretical consideration of the potential positive contribution of reabsorption.  When this is better understood, consideration can then be given to the integration of the pumping mechanism with a method for replacing the source condensates to produce a continuously pumped atom laser.



