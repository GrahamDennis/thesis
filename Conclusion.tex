\chapter{Conclusion}
\label{Conclusion}
\graphicspath{{Figures/Conclusion/}{Figures/Common/}}

The coherent atomic sources considered in this thesis cannot compete with the flux of present thermal sources.  They do, however, have two potentially significant advantages over thermal sources for atom interferometry.

The first is their narrower linewidth, and therefore velocity spread.  This enables the more efficient operation of large momentum-transfer processes (which are highly velocity selective).  Large momentum-transfer processes are important for increasing the interaction time of the atoms, and therefore the phase shift to be measured.  Also, for larger interaction times the expansion of the atomic source increases signal loss \citep{Dimopoulos:2007uq}, an effect which is reduced by the use of a condensed source.

The second advantage of coherent sources is their potential for producing squeezed and entangled atomic sources.  In the context of quantum optics, squeezing has been used to reduce the shot noise limit (the standard quantum limit $\propto 1/\sqrt{N}$ where $N$ is the number of particles observed) in precision measurement, while entanglement has been used to demonstrate measurement precision scaling at the fundamental Heisenberg limit $\propto 1/N$ \citep{Leibfried:2004}.  Squeezed and entangled coherent atomic sources would offer the same potential advantages for atom interferometry.  Moreover, such non-classical states of matter are interesting for fundamental tests of quantum mechanics.



For the optical pumping mechanism considered in \chapterref{OpticalPumping}, the achievable flux of the atom laser is limited by the rate at which condensates can be produced to replace the source condensate as it is depleted.  Even if all the atoms were transferred to the lasing condensate with perfect efficiency, the continuous atom laser produced could never exceed the average flux with which the source condensates can be produced.  The optical pumping experiment described in that chapter is therefore not a useful source for atom interferometry in its present form.  However, there is some evidence that the harmful reabsorption processes in that experiment may be suppressed due to a quantum-mechanical interference effect.  A better theoretical understanding of this process would determine whether or not the source condensate could be replaced with a thermal source.  If possible, this would greatly increase the potential flux of the produced atom laser.

The evaporatively-driven pumping mechanism considered in \chapterref{KineticTheory}, is supplied by a thermal source and therefore it could be possible for it to provide a condensed source of atoms at a similar flux to the thermal source.  However, the actual flux is necessarily lower as the evaporation process removes atoms from the system as part of the pumping mechanism.  One of the aims of \chapterref{KineticTheory} was to determine just how much lower the flux of the condensed source would be.  It was found that the largest flux achievable from an available thermal atomic source was $\sim\unit[10^5]{s\textsuperscript{-1}}$, an average flux that is also achievable by producing independent condensates.

While neither of the proposed pumped atom lasers would be competitive with thermal sources for atom interferometry, pumping could improve other atom laser experiments.  In particular, experiments in which squeezing or entanglement of atomic beams is being produced and measured.  In these experiments, the atomic beam will be condensed as it would be more difficult to detect squeezing or entanglement with a thermal source.  For these experiments, a pumped atom laser could potentially offer a narrower linewidth, increased flux, and continuous operation, all of which would increase the sensitivity of measurements of atomic squeezing and entanglement.

\chapterref{Peaks} discussed one such process that leads to the formation of entangled atomic beams.  Although in the experiment discussed in that chapter it would only be possible to measure number-difference squeezing and correlations in atom detection events, a different experiment was proposed in which it may be possible to measure the entanglement of the produced atomic beams.  This chapter has also demonstrated the excellent agreement that is achievable between theory and experiment for BEC systems.  This is possible due to extraordinary degree of control over external noise sources that is possible in experiments, and the effectiveness of the theoretical techniques available for modelling them.

Here is where I believe I'm missing a final paragraph.

% This thesis documents the work of a theorist working in close collaboration with experimentalists.  This is particularly evident in \chapterref{Peaks} and \chapterref{OpticalPumping}, but it also been the case for work that has not been documented in this thesis.  Attempting to resolve the minor mysteries that sometimes arise in experimental physics, and working closely with the experimentalists as they are found, cause confusion, and finally resolved.

% Now for a strong finish.


% Contrary to what has been expected previously, this chapter has demonstrated that optical pumping of an atom laser is feasible without making the condensate so narrow as to make it essentially transparent in at least one dimension.  The next step in the investigation of this pumping process must be the theoretical consideration of the potential positive contribution of reabsorption.  When this is better understood, consideration can then be given to the integration of the pumping mechanism with a method for replacing the source condensates to produce a continuously pumped atom laser.



