\chapter{Quantum kinetic theory model of a continuous atom laser}
\label{KineticTheory}
\graphicspath{{Figures/KineticTheory/}{Figures/Common/}}

Continuous pumping of an atom laser is a key tool for producing superior atomic sources. Besides the obvious benefit of higher flux, it also promises improved modal stability and linewidth, much as it does for the optical laser.

The pumping process of an atom laser --- just like that of an optical laser --- is a necessarily irreversible process. This irreversibility enters through the coupling of the lasing mode to a much larger system (the reservoir). For the optical laser this reservoir is comprised of the (almost) empty modes of the de-excited atomic field of the atoms driving the lasing process. In the case of the atom laser, there are two possible choices for the reservoir providing the irreversibility: empty modes of an atomic field, or empty modes of an optical field. The former case was considered in the last chapter; the latter is the subject of this chapter\footnote{FIXME: This intro needs rewriting, but it is impossible to do so without both \chapterref{OpticalPumping} having been written and the appropriate part of the Literature Review / Background Theory chapters.}. The results presented in this chapter have been submitted for publication\footnote{FIXME: Do this.}. The results and analysis presented in \sectionref{KineticTheory:Results} of this chapter was my own work. The model presented in \sectionref{KineticTheory:Model} is based on prior work \citep{Davis:2000vn,Bijlsma:2000}. The derivation of the three-body loss term in \sectionref{MethodsAppendix:QKT3BodyLoss} and the code the results in this chapter are based on are the work of \emph{Matthew Davis}.

\section{Motivation}

Continuous pumping of an atom laser is a key tool for producing superior atomic sources. Besides the obvious benefit of higher flux, it also promises improved modal stability and linewidth, much as it does for the optical laser.

There are two essential steps towards the continuous pumping of an atom laser. The first is a delivery system for filling an atomic reservoir with ultracold atoms. The second is a process that causes at least some of those atoms to make an irreversible, atom-stimulated transition into the BEC.

Continuous delivery of ultracold atoms has been demonstrated in a number of experiments\footnote{FIXME: Add citations.}, and is an important component of thermal atomic interferometry experiments. FIXME: Complete paragraph.

The atom-stimulated transitions into the condensate can be made irreversible by coupling to a reservoir. There are two possible reservoirs: the empty modes of the electromagnetic field accessible via a transition from an excited atomic state, or the empty modes of the atomic field accessible via evaporation. Having considered the former in \chapterref{OpticalPumping} it is the latter under consideration in this chapter.

Sequential reloading of a target BEC was achieved using optical tweezers \citep{Chikkatur:2002qa}, where a series of source condensates were added adiabatically by manipulating the trapping potentials, and excitations were subsequently removed by continuous evaporation. This milestone experiment maintained the condensate fraction, and therefore the potential flux of a potential atom laser. An atom laser produced from such an experiment would, however, not possess the desired narrow linewidth as the source condensates used were of a similar size to or larger than the condensate being replenished causing significant scattering into modes other than the target condensate. To produce an atom laser with a narrow linewidth it would be necessary for the atomic source to negligibly disrupt the target condensate. While this could be achieved by merging the target condensate with significantly smaller condensates more frequently, it is technically very challenging to develop high flux sources of Bose-condensed atoms compared to sources at higher temperature, which have a higher average flux. In this chapter it is shown that a similar experiment using an ultra-cold \emph{thermal} source ought to be able to pump the target BEC and maintain a significant BEC population using a phase-preserving Bose-enhanced process.

Past work: \citet{Proukakis:2003}.

%This method has the advantage that it can be performed without the presence of resonant light, but the obvious disadvantage that it relies on the system approaching thermal equilibrium, and will therefore be reversed by the addition of atoms above the condensate temperature. In this chapter it is demonstrated that a driven system undergoing evaporative cooling can produce a high-flux, phase-stable atom laser for a range of experimental parameters.

\section{Scheme}
\label{KineticTheory:Scheme}

\begin{figure}
    \centering
        \includegraphics[width=10cm]{QKTScheme}
    \caption{Schematic of the experimental setup.}
    \label{KineticTheory:QKTScheme}
\end{figure}

The proposed scheme for a pumped atom laser is illustrated in \figureref{KineticTheory:QKTScheme} and is very similar to the processes used to evaporate a thermal cloud to condensation in a magnetic trap and produce a (quasi-continuous) atom laser. The additional element in this scheme is a process for replenishing the cloud of thermal atoms in the trap.  

In this scheme the gain process for the condensate is the same Bose-enhanced scattering between thermal atoms and the condensate that drives condensate growth when evaporating to produce condensate \citep{Gardiner:1997kx,Davis:2000vn,Bijlsma:2000}.  This process becomes irreversible when one of the scattered atoms has enough energy to cross the evaporation surface and be removed from the thermal cloud.  The loss of atoms from the thermal cloud is balanced by a replenishment process that couples the thermal cloud to a source of atoms at finite temperature.

The atom laser beam itself is to be produced by large momentum-transfer Raman outcoupling from the condensate to minimise the loss from the atom laser as it traverses the evaporation surface\footnote{An alternative configuration would be to place the evaporation surface \emph{above} the thermal cloud, however the effectiveness of the evaporation surface is reduced as its contact area with the thermal cloud is reduced. In a typical trap with a trapping frequency of $\omega = 2\pi \times \unit[100]{Hz}$ in the vertical direction and an aspect ratio of $10$, the separation between the centre of the condensate and the centre of the magnetic trap and the evaporation surface is $y_\text{sag} = \unit[25]{\micro m}$.  For a thermal cloud of $N=10^6$ rubidium atoms at the critical temperature, the $1/e$ radius in the vertical direction is $y_\text{thermal} = \unit[15]{\micro m}$. Placing the evaporation surface above the thermal cloud would then significantly reduce its contact area with the thermal cloud. For this reason the evaporation surface in \figureref{KineticTheory:QKTScheme} has been pictured below the thermal cloud.}. To minimise direct outcoupling from the thermal cloud, the two Raman lasers used in the outcoupling process can be focussed to only intersect in the immediate vicinity of the condensate.

A dynamic equilibrium will be reached when the rate of atom loss from the condensate due to outcoupling balances the rate of atoms gained due to scattering with the thermal cloud.  If the evaporative surface is tuned so that atoms of energy $\varepsilon_\text{cut}$ and higher are rapidly and continually removed from the trap, then all collisions that give atoms energy greater than $\varepsilon_\text{cut}$ will become irreversible. As $\varepsilon_\text{cut}$ is lowered, a larger fraction of the scattering processes that leave atoms in the condensate mode will become irreversible. This suggests that there must be some value of $\varepsilon_\text{cut}$ for which the condensate experiences net gain. What is not clear is whether the net gain can proceed efficiently, i.e. on a timescale much shorter than other losses from the condensate.  Lowering $\varepsilon_\text{cut}$ also reduces the total number of thermal atoms present. In the limit that $\varepsilon_\text{cut}$ reaches the condensate energy, there will be no background gas at all, and the condensate cannot experience net gain.  We therefore expect that for a given set of parameters, there will be an optimal value for $\varepsilon_\text{cut}$ that maximises the net gain, which may or may not be positive. In order to examine this issue, quantum kinetic theory (QKT) \citep{Gardiner:1997tz,Jaksch:1997ug,Gardiner:1998wx,Jaksch:1998sj,Gardiner:2000ug,Lee:2000vs,Davis:2000vn} has been employed, which has been effective in describing the growth of condensates \citep{Davis:2000vn}.

\section{Model}
\label{KineticTheory:Model}

\begin{figure}
    \centering
        \includegraphics[width=10cm]{QKTModel2}
    \caption{Schematic of the theoretical model.}
    \label{KineticTheory:QKTModel}
\end{figure}

The theoretical model described in this section is an extension of the kinetic model of \citet{Bijlsma:2000}, which was successfully used to study condensate growth in an experiment in which a cloud of thermal atoms just above condensation temperature were shock-cooled below transition \citep{Miesner:1998}.  After shock-cooling the atoms were left to equilibrate, with condensate formation being driven by the same collisional processes that would drive condensate growth in the proposed pumped atom laser experiment described in the previous section.  To fully describe this proposed experiment, the kinetic model of \citeauthor{Bijlsma:2000} must be modified to include the effects of the replenishment and outcoupling processes illustrated in \figureref{KineticTheory:QKTScheme}. 

Another important process that must be included in the model is three-body recombination, which is the dominant loss process in typical BEC experiments \citep{Burt:1997fk,Soding:1999}. Without the inclusion of this process, for given replenishment and outcoupling rates the largest condensate would be formed in the absence of evaporation as outcoupling from the condensate would be the only loss process in the system. In fact, this condensate number would be independent of the temperature of the replenishment source, depending only on the flux of atoms delivered to the system and the outcoupling rate from the condensate. This unphysical result is because in the absence of a density-dependent loss process, simply increasing the density is a feasible method of approaching degeneracy. It would be possible to reach condensation with room-temperature atoms in a harmonic trap simply by confining enough atoms! To avoid such unphysical results, the effect of three-body loss as the dominant density-dependent loss process must be included in the model.

\parasep

The starting point of the kinetic model presented here is to treat separately the thermal and condensed components of the system in \figureref{KineticTheory:QKTScheme}.

The condensed component is assumed to be a quantum fluid obeying a Gross-Pitaevskii-type equation, however we make a further approximation and assume that the condensate is sufficiently occupied that it has a Thomas-Fermi profile.  The condensate dynamics are then fully described by the number of condensed atoms $N_0(t)$.

The thermal cloud is assumed to be well described within the Hartree-Fock approximation \citep[Chapter 8]{PethickSmith} as comprised of particle-like excitations moving in the effective potential of the harmonic trap plus condensate mean field.  To reduce the dimensionality of the full phase-space distribution function for the thermal cloud $f(\bm{r}, \bm{p}, t)$, it is assumed that the system is ergodic, i.e.  that all points in the phase space having the same energy are equally probable.  Under this approximation, the thermal cloud is then described by its energy distribution function $g(\varepsilon, t)$ and the density of states $\rho(\varepsilon, t)$.  The assumption of ergodicity has been shown in the past to give good agreement with experiment when asymmetric spatial or momentum dynamics are not significant \citep{Bijlsma:2000,Davis:2000vn}.  Note that the time-dependence of the density of states $\rho(\varepsilon, t)$ comes from the contribution of the condensate mean field to the effective potential experienced by the thermal atoms.


As the model presented here is very similar to that presented in \citep{Bijlsma:2000} with some additional terms, a derivation of the common terms is omitted. As a summary, the derivation proceeds by taking a semiclassical Boltzmann equation for the phase-space distribution function of the thermal cloud $f(\bm{r}, \bm{p}, t)$ including collisional terms and using the ergodic approximation to obtain an equation of motion for the energy distribution function $g(\varepsilon, t)$. This equation is self-consistently matched with a Gross-Pitaevskii equation for the condensate before making the Thomas-Fermi approximation to obtain an equation of motion for the number of condensed atoms $N_0(t)$. An example application of this method to derive the appropriate terms for three-body loss is given in \sectionref{MethodsAppendix:QKT3BodyLoss}. Further, a detailed discussion of this theory is given in the review article \citep{Proukakis:2008}.

Separating the contributions of the different processes involved, the equations of motion for the model for a collision-driven pumped atom laser considered here are
\begin{align}
    \frac{d N_0}{d t} =\begin{split}
        &\relphantom{+}\left. \frac{d N_0}{d t}\right|_\text{thermal--condensate} \\
        &+\left. \frac{d N_0}{d t}\right|_\text{3-body loss} \\
        &+\left. \frac{d N_0}{d t}\right|_\text{outcoupling}
    \end{split},
    & \frac{\partial (\rho g)}{\partial t} = \begin{split}
        &\relphantom{+}\left. \frac{\partial (\rho g)}{\partial t}\right|_\text{thermal--thermal} \\
        &+\left. \frac{\partial (\rho g)}{\partial t}\right|_\text{thermal--condensate} \\
        &+\left. \frac{\partial (\rho g)}{\partial t}\right|_\text{3-body loss} \\
        &+\left. \frac{\partial (\rho g)}{\partial t}\right|_\text{replenishment} \\
        &+\left. \frac{\partial (\rho g)}{\partial t}\right|_\text{redistribution}
    \end{split},
    \label{KineticTheory:EvolutionEquations}
\end{align}
where the subscripts `thermal--thermal' and `thermal--condensate' denote Bose-enhanced collisional processes between atoms in the corresponding states (\figureref{KineticTheory:ProcessDiagrams}(a) and (b), respectively), the subscript `3-body loss' indicates the contribution due to three-body recombination, the subscript `replenishment' indicates the contribution due to the replenishment of the thermal cloud (\figureref{KineticTheory:ProcessDiagrams}(d)), the subscript `outcoupling' indicates the contribution due to outcoupling from the condensate to form the atom laser (\figureref{KineticTheory:ProcessDiagrams}(e)), and the subscript `redistribution' indicates the contribution due to the redistribution of population in energy space due to the changes of the energies of the occupied levels as the mean-field of the condensate changes (\figureref{KineticTheory:ProcessDiagrams}(c)). It is assumed that atoms with energy greater than the evaporative energy cut-off $\varepsilon_\text{cut}$ are removed from the system sufficiently quickly that  $g(\varepsilon > \varepsilon_\text{cut}) = 0$.

\begin{figure}
    \centering
    \includegraphics[width=14cm]{ProcessDiagrams}
    \caption{Schematic of processes involved in the evolution of the kinetic model described by \eqref{KineticTheory:EvolutionEquations}.  The upper shaded rectangle in each subfigure represents the energy distribution function $g(\varepsilon, t)$ of the thermal cloud, and the bottom dark blue rectangle represents the condensate with occupancy $N_0(t)$ and energy $\varepsilon = \mu(t)$. Figures (a) and (b) represent collisional processes involving two thermal atoms and one thermal and one condensate atom respectively. Figure (c) represents the change in the energy distribution function $g(\varepsilon, t)$ if the condensate occupation (and hence chemical potential) changes, changing the energies of every energy level. Figure (d) represents the replenishment of the thermal cloud from an atomic reservoir, and Figure (e) represents outcoupling from the condensate mode to produce the atom laser.}
    \label{KineticTheory:ProcessDiagrams}
\end{figure}

The forms of the `thermal--thermal', `thermal--condensate' and `redistribution' terms in \eqref{KineticTheory:EvolutionEquations} are given in \sectionref{MethodsAppendix:QKTOtherTerms} and derivations are given in \citep{Bijlsma:2000}.

The outcoupling process from the condensate is modelled as a simple linear loss process with corresponding rate constant $\gamma$,
\begin{align}
    \left.\frac{d N_0}{d t}\right|_\text{outcoupling} &= - \gamma N_0.
    \label{KineticTheory:OutcouplingProcess}
\end{align}
In modelling the outcoupling in this way, any outcoupling from thermal modes has been neglected. As discussed in \sectionref{KineticTheory:Scheme}, this is a reasonable approximation if focused Raman lasers are used for the outcoupling which only intersect in the immediate vicinity of the condensate. 

The thermal cloud is modelled as being continuously replenished from a source that provides a constant flux $\Phi$ of atoms at a temperature $T$.  To avoid tying the model to any particular replenishment mechanism, we assume a best-case scenario in which each energy level $\varepsilon$ in the source is coupled directly to the level in the thermal cloud with the same energy above the condensate chemical potential $\mu(t)$, i.e. the lowest energy level of the source ($\varepsilon=0$) is coupled directly to the lowest energy level in the trap ($\varepsilon = \mu(t)$).  This simple model gives the form of the contribution due to replenishment as
\begin{align}
    \left. \frac{\partial \big(\rho(\varepsilon, t) g(\varepsilon, t))}{\partial t} \right|_\text{replenishment} &= \Gamma \rho_0(\varepsilon - \mu(t)) g_T(\varepsilon - \mu(t)),
    \label{KineticTheory:ReplenishmentProcess}
\end{align}
where $\rho_0(\varepsilon)$ is the density of states in the absence of a condensate, $g_T(\varepsilon)$ is the Bose-Einstein  distribution at temperature $T$, and $\Gamma$ is a rate constant such that
\begin{align}
    \Gamma \int_0^\infty \rho_0(\varepsilon) g_T(\varepsilon)\, d\varepsilon = \Phi,
    \label{KineticTheory:GammaPhiRelation}
\end{align}
where $\Phi$ is the flux of atoms from the source \emph{before} evaporation.  The derivation of the contributions to \eqref{KineticTheory:EvolutionEquations} due to three body loss were performed by \emph{Matthew Davis}, and are given in \sectionref{MethodsAppendix:QKT3BodyLoss}.
%While this is perhaps an unreasonable assumption, it means that we can easily determine the steady state of the system.

\parasep

We summarise here the approximations made in obtaining the kinetic model \eqref{KineticTheory:EvolutionEquations}:
\begin{enumerate}[(i)]
    \item The energy scale of the thermal cloud is large enough that all excitations are particle-like and not collective excitations such as phonons. Phonon-like excitations are only important for particle energies $\varepsilon \lesssim 2\mu(t)$ \citep[\S 8.3.1]{PethickSmith}. Hence, we require the that the energy scale for the thermal cloud $\varepsilon_\text{cut}$ be much larger than $\mu(t)$.
    \item The phase-space distribution of the thermal cloud is ergodic and hence is purely a function of energy. This assumption is true at equilibrium, however it needs some justification when used in non-equilibrium scenarios. In this case, asymmetric behaviour of the condensate is not expected in either position or momentum space 
    \item The condensate density is sufficiently large that it is well-described by a Thomas-Fermi profile. This approximation is justified as it is only the large-condensate limit that is of interest as a large condensate will be necessary for the production of a high-flux atom laser in this scheme. In making this approximation, the effects of both the normal and anomalous densities of the thermal cloud on the condensate have also been neglected.
    \item Evaporation occurs on a time-scale faster than collisions. This is the usual requirement during evaporation to condensation, and so should be satisfied in the proposed experiment.
\end{enumerate}

The computer code used to solve the kinetic model \eqref{KineticTheory:EvolutionEquations} was written by \emph{Matthew Davis}. The results and analysis presented in the remainder of this chapter are my own work.

\section{Results}
\label{KineticTheory:Results}

For a given trap geometry, the model is fully defined by the flux of replenishment atoms $\Phi$, the temperature $T$ of those atoms, the energy of the evaporative cut $\varepsilon_\text{cut}$, and the outcoupling rate from the condensate $\gamma$. In this section the results of the kinetic model for some `typical' parameter values are presented, and the dependence of the model on each of the parameters is examined.

Our numerical simulations are based on a trap and conditions similar to that of \citep{Kohl:2002}, who precooled a cloud of $\nucl{87}{}{Rb}$ atoms to an initial temperature greater than the critical temperature, before performing evaporative cooling to study condensate growth. The trap in the experiment was axially-symmetric with radial and axial trapping frequencies of  $\omega_\rho = \unit[2 \pi \times 110]{Hz}$ and $\omega_z = \unit[2\pi \times 14]{Hz}$ respectively.

To numerically solve the kinetic model, \eqref{KineticTheory:EvolutionEquations} is discretised along the energy dimension and the resulting coupled differential equations are solved with an adaptive fourth-fifth Runge-Kutta \citep{NumericalRecipes} method. Our results are mainly concerned with the steady-state of the kinetic model, which we define as being reached when the condensate number has changed by less than $0.1\%$ or $1$ atom in $\unit[100]{ms}$.  The initial state for the simulation is chosen to be a truncated Bose-Einstein distribution containing (before truncation) $N_\text{initial} = \unit[4.2\times 10^6]{atoms}$ at the same temperature as the replenishment reservoir.  This state is chosen as a representation of the steady-state of the system prior to evaporation.  In the trap considered, the critical temperature for $4.2\times 10^6$ atoms is $T_c = \unit[340]{nK}$.


\subsection{Typical results and parameter studies}
\label{KineticTheory:ParameterStudies}

\begin{figure}
    \centering
    \includegraphics[width=15cm]{EnergyDistributionFunctionEvolution}
    \caption{Results of the kinetic model for $\Phi = \unit[8.4 \times 10^5]{atoms/s}$, $T=\unit[540]{nK}$, $\varepsilon_\text{cut} = 3 k_B T \approx 610 \hbar \overline{\omega}$, and $\gamma = \unit[0.3]{s\textsuperscript{-1}}$. Figure (a) highlights the dynamics of the occupation of the thermal energy levels for $t < \unit[0.5]{s}$, while (b) illustrates the equilibration of the total and condensed atom numbers over $\sim \unit[10]{s}$. The energy distribution at $t=0$ is a truncated Bose-Einstein distribution containing (before truncation) $N=4.2\times 10^6$~atoms at $T=\unit[540]{nK}$.
    }
    \label{KineticTheory:EnergyDistributionFunctionEvolution}
\end{figure}

As a depiction of the `typical' time-dependence of the results obtained from the kinetic theory model \eqref{KineticTheory:EvolutionEquations}, we consider the case of pumping an atom laser continuously with a source such that the initial number $N = 4.2\times 10^6$ is transferred to the system once every 5 seconds giving a flux of $\Phi = \unit[8.4 \times 10^5]{atoms/s}$.  The temperature of the replenishment source is chosen to be $T=\unit[540]{nK}$, 60\% above the condensation temperature of the system before evaporation.  For the remaining model parameters, we choose the evaporative cut-off to be $\varepsilon_\text{cut} = 3 k_B T$, and the outcoupling rate from the condensate to be $\gamma = \unit[0.3]{s\textsuperscript{-1}}$.  

\figureref{KineticTheory:EnergyDistributionFunctionEvolution} illustrates the results of the simulation of this system.  \figureref{KineticTheory:EnergyDistributionFunctionEvolution}(a) shows the energy distribution of the thermal cloud cooling from the initial truncated Bose-Einstein distribution to a distribution with a lower average energy per particle.  \figureref{KineticTheory:EnergyDistributionFunctionEvolution}(b) demonstrates that despite pumping the system with an atomic reservoir above critical temperature that it is possible to reach a steady-state distribution in which the condensate is macroscopically occupied.  In this example, the steady-state condensate fraction is 33\%.

\begin{figure}
    \centering
    \includegraphics[width=15cm]{ParameterStudies}
    \caption{FIXME: The graphic needs changing as not all of the data is present yet. Caption will then need updating.}
    \label{KineticTheory:ParameterStudies}
\end{figure}

The details of the equilibration of the system are not the subject of investigation here, instead our interest is in the equilibrium itself, and in determining the feasibility of creating a pumped atom laser driven by a non-condensed atomic source.  As a first step towards this investigation we consider the dependence of the equilibrium condensate number on the parameters of the system: $\Phi$, $T$, $\varepsilon_\text{cut}$, and $\gamma$. \figureref{KineticTheory:ParameterStudies} displays the dependence of the equilibrium of the results pictured in \figureref{KineticTheory:EnergyDistributionFunctionEvolution} on the parameters of the model.

The majority of the parameter dependences depicted in \figureref{KineticTheory:ParameterStudies} are trivial; increasing the relevant parameter causes a monotonic change in the equilibrium condensate number.  Increasing the flux of atoms to the system increases the equilibrium condensate number (\figureref{KineticTheory:ParameterStudies}(a)), while increasing the temperature of the replenishment source or increasing the outcoupling rate reduces the equilibrium condensate number (\figureref{KineticTheory:ParameterStudies}(b) and \figureref{KineticTheory:ParameterStudies}(d) respectively).  The only non-trivial behaviour is displayed by \figureref{KineticTheory:ParameterStudies}(c) in which the dependence on the evaporative cut-off $\varepsilon_\text{cut}$ is illustrated.  For large evaporative cut-offs, few atoms will be lost due to evaporation and the system will reach equilibrium when the flux of atoms into the system is balanced by three-body losses from the condensate.  As $\varepsilon_\text{cut}$ is reduced, more atoms are lost due to evaporation and the mean energy per particle reduces causing the condensate size to increase.  As $\varepsilon_\text{cut}$ continues to reduce, an increasing fraction of the replenishment atoms have an energy greater than $\varepsilon_\text{cut}$ causing a lower effective atomic flux to be delivered to the system, hence reducing the potential size of any condensate formed.  These two competing effects are the origin of the existence of an optimum equilibrium condensate number as a function of $\varepsilon_\text{cut}$ in \figureref{KineticTheory:ParameterStudies}(c).

As discussed earlier, in the absence of three-body loss the equilibrium condensate number would continue to increase as $\varepsilon_\text{cut}$ is increased, which would lead to the unphysical conclusion that evaporating \emph{reduces} the equilibrium condensation number.  This is demonstrated by the dashed line in \figureref{KineticTheory:ParameterStudies}(c) which asymptotes towards $N=\Phi/\gamma = \unit[2.8\times 10^6]{atoms}$ in the limit $\varepsilon_\text{cut}\rightarrow \infty$.  The primary effect of three-body loss in the model is to significantly increase losses for large numbers of atoms in the system causing the monotonic dependence on $\varepsilon_\text{cut}$ to give way to the existence of an optimal $\varepsilon_\text{cut}$ where an optimum trade-off is made between atomic losses due to evaporation and three-body loss. Despite three-body loss being included in the simulations for the remaining panels of \figureref{KineticTheory:ParameterStudies} the equilibrium condensate number depends monotonically on the corresponding parameters. FIXME: Another comment or two is needed to explain that this monotonic dependence in the presence of three-body loss for the remaining parameters is expected. (Check in the extremes that these behaviours are maintained).

%The replenishment process could be treated in a best-case scenario in two different ways: coupling each energy levels directly that have the same energy, or coupling the nth level in one system to the nth level in the other system. The first case corresponds to rapidly combining the systems on a timescale short enough that neither system has time to react; the second case corresponds to the opposite limit in which the systems are coupled slowly enough that each energy level undergoes an adiabatic change towards the corresponding level in the other system.

% We still need more thermal sources for the high-temperature limit part of the text.


\subsection{Behaviour in the high-temperature limit}
In the previous section, the dependence of the equilibrium condensate number on the model parameters was investigated. The physical question that we desire to address with this model is what are the requirements on the replenishment source to produce a pumped atom laser with favourable properties for precision measurement?

Although it would be possible to create a pumped atom laser by combining condensates in a manner similar to the experiment by \citet{Chikkatur:2002qa}, such an atom laser would have significantly reduced phase-stability unless the replenishment process were essentially continuous. However, to replenish a condensate by collisional interactions with a continuous source of condensed atoms, the replenishment source would itself need to have many of the desired properties of a pumped atom laser! Instead, it would be preferable to use a source \emph{above} condensation temperature for replenishment.

We consider now the experimentally-relevant limit of replenishing the thermal cloud using a high-flux source of thermal atoms. For such sources two simplifications are possible. First, for temperatures greater than $T_c$ the Bose-Einstein energy distribution of the source $g_T(\varepsilon)$ is well approximated by the Boltzmann distribution $g_T(\varepsilon) \approx \zeta e^{-\beta \varepsilon}$ for some constant $\zeta$, and $\beta = \left(k_B T\right)^{-1}$. Secondly, for high temperature sources the optimum evaporation cut-off $\varepsilon_\text{cut}$ will be much smaller than the characteristic energy of the source $k_B T$, and hence $\displaystyle \eta = \frac{\varepsilon_\text{cut}}{k_B T} \ll 1$.  From these simplifications it can be seen that the energy distribution below the evaporation cut-off is well described by the single parameter $\zeta$ as $g_T(\varepsilon \leq \varepsilon_\text{cut}) \approx \zeta$.

At this point, no overall simplification has occurred as we have simply rewritten the temperature dependence of the replenishment source in terms of the parameter $\zeta$.  However, as the energy distribution of the replenishment source only affects the kinetic model through \eqref{KineticTheory:ReplenishmentProcess}, its influence on the system dynamics can only be through the combined quantity $\kappa = \Gamma \zeta$. An expression for $\kappa$ directly in terms of relevant experimental quantities can be obtained using the definition \eqref{KineticTheory:GammaPhiRelation},
\begin{align}
    \Phi &= \Gamma \int_0^\infty \rho_0(\varepsilon) g_T(\varepsilon)\, d\varepsilon\\
    &= \Gamma \int_0^\infty \frac{\varepsilon^2}{2 (\hbar \overline{\omega})^3} \zeta e^{-\beta \varepsilon}\, d\varepsilon\\
    &= \Gamma \zeta \frac{1}{2 (\hbar \overline{\omega})^3} \int_0^\infty \varepsilon^2 e^{-\beta \varepsilon}\, d\varepsilon\\
    &= \left(\frac{k_B T}{\hbar \overline{\omega}}\right)^3 \Gamma \zeta\\
    \kappa &\equiv \Gamma \zeta = \Phi \left(\frac{\hbar \overline{\omega}}{k_B T}\right)^3
\end{align}
where $\overline{\omega} = \left(\omega_x \omega_y \omega_z\right)^{\frac{1}{3}}$ is the geometric mean of the trapping frequencies, and $\displaystyle\rho_0(\varepsilon) = \frac{\varepsilon^2}{2 (\hbar \overline{\omega})^3}$ is the density of states in a harmonic trap in the absence of a condensate \citep{PethickSmith}.

The quantity $\kappa$ is a figure-of-merit for the thermal source. It quantifies the qualitative behaviour already known: for the same atomic flux $\Phi$, a source with a lower temperature will result in a larger condensate (FIXME: see some figure); and for the same temperature, a source with a higher atomic flux will also result in a larger condensate (FIXME: see some other figure). Our interest is in determining what values of $\kappa$ are necessary to produce a pumped atom laser, and whether such values are achievable.

For the limit of high-temperature atomic sources, we have reduced the four variables $(\Phi, T, \varepsilon_\text{cut}, \gamma)$ required to define the model \eqref{KineticTheory:EvolutionEquations} down to three $(\kappa, \varepsilon_\text{cut}, \gamma)$. Of these three, our main interest is in the dependence of the system on the properties of the atomic source through $\kappa$. In contrast, the dependence of the equilibrium condensate number on the outcoupling rate $\gamma$ is well understood and the results would not be expected to change qualitatively with $\gamma$. It is therefore appropriate to choose a representative value for the outcoupling rate (here $\gamma = \unit[0.3]{s\textsuperscript{-1}}$) and focus on the remaining two quantities.

As discussed in the previous section, there is an optimal choice for the evaporation cut-off $\varepsilon_\text{cut}$. Our interest here is in the best-case scenario: for a given thermal source, what is the largest condensate we can produce? To examine this question and to verify that $\kappa$ does fully describe the properties of the thermal source in the appropriate limit we have performed a parameter scan of the model \eqref{KineticTheory:EvolutionEquations} for a range of fluxes $\Phi$ and temperatures $T$ of the atomic source, for each combination determining the optimum evaporative cut $\varepsilon_\text{cut}$ to give the largest steady-state condensate number. The results of this parameter study are displayed in \figureref{KineticTheory:FigureOfMerit}.

\begin{figure}
    \centering
    \includegraphics[width=15cm]{FigureOfMerit}
    \caption{FIXME: Figure caption. Plot of equilibrium $N_0$ against $\kappa$ for $\eta < 0.1$ (black), $0.1 < \eta < 0.5$ (red) and $\eta > 0.5$ (blue). Or something like that.}
    \label{KineticTheory:FigureOfMerit}
\end{figure}

\begin{figure}
    \centering
    \includegraphics[width=7cm]{CondensateFraction}
    \caption{FIXME: Figure caption. Plot of equilibrium condensate fraction against $\kappa$ for the usual suspects.}
    \label{KineticTheory:CondensateFraction}
\end{figure}

and to verify that $\kappa$ does fully describe the properties of the thermal source in the appropriate limit we have performed a parameter scan over the space $(\Phi, T, \varepsilon_\text{cut})$, choosing $\gamma = \unit[0.3]{s\textsuperscript{-1}}$ as noted above.



There are a number of high-flux atomic sources that have been developed both for the production of BEC and, more recently, in the consideration of atom-interferometry using cold thermal atomic beams. Such `cold' atomic sources range from Zeeman slowers, MOT's, LVIS configurations, etc. The aim of this chapter is to examine the feasibility of the use of such thermal atomic sources in the production of high-flux continuous, degenerate atomic sources. Studying this limit

Creating

The most experimentally interesting case

The most experimentally rele


Talk about various high-flux atomic sources that could be used. If it weren't for three-body loss, it wouldn't much matter and we could just do whatever. But. 


The limit that is of the greatest experimental relevance is that of replenishing a condensate with high-flux sources well above condensation temperature.

high-flux sources


To facilitate extracting useful information from this model, we consider the experimentally relevant limit of high-flux replenishment sources far above condensation temperature. Although a complete analytic model cannot be obtained in this limit, we can recognise an equivalence between difference sources. For such high-temperature sources, the Bose-Einstein distribution is well approximated by the Boltzmann distribution $g_B(\varepsilon) = A e^{-\beta\varepsilon}$ for some constant $A$, and $\beta = \left(k_B T\right)^{-1}$. For such high-temperature sources, the optimum choice of the evaporation cut-off $\varepsilon_\text{cut}$ will be small compared to stuff. This will mean that $\eta$ will be small. If $g_B(\varepsilon_\text{cut}) = A e^{-\eta} \approx g_B(0)$, then the thermal source is completely described by the parameter $K=\Phi A$.




In the limit of a reservoir of high-temperature atoms, we can make some useful statements and comparisons.


In the limit of high-temperature 


This is the main contribution of the paper. We study the behaviour in the high temperature limit, obtain a simple model, verify the model and then compare against experiment.




\section{Conclusion}
Thermal sources must be close to degenerate to be useful. And even then they aren't that useful.