\chapter*{Abstract}
%\addcontentsline{toc}{chapter}{Abstract}
\graphicspath{{Figures/Common/}}


Measurement is our fundamental tool for learning about the world around us.  It is from observing trends in measurements that we develop the theories that enable us to predict future behaviour, and it is against measurements that we determine the validity of these theories.  Increases in the precision of our measurements are fundamental to our understanding.

Atom interferometry is a new method for performing precision measurements that uses the matter-wave nature of atoms to perform interferometry experiments analogous to those performed with photons.  However, in contrast to optical interferometry, which uses coherent sources of photons, atom interferometry uses thermal atomic sources, in part due to the unavailability of high-flux coherent atomic sources.  Although pulsed coherent atomic sources are presently available, continuous sources are not.  Creating a truly continuous coherent source for atoms is tricker than for photons.  One of the largest challenges is that atom number is conserved.  A source of atoms is therefore necessary to produce a truly continuous atom laser.  This source must be used to replenish the lasing mode of the atom laser, and the process must operate without significantly disturbing the coherence properties of the lasing mode.  It is this replenishment or pumping process that has been investigated theoretically in this thesis.

There are only two choices for the reservoir that makes the replenishment (or pumping) process of an atom laser irreversible: the empty modes of the optical field, and the empty modes of the atomic field.  Processes of both forms are considered.  Using an optical reservoir has the advantage that atoms are not necessarily lost as part of the pumping process, which is necessary when using an atomic reservoir.  The efficiency of processes using an optical reservoir can therefore be higher.  Using an atomic reservoir, however, has the advantage that it is easier to implement as one can use the standard experimental technique of evaporation which is commonly used in the production of pulsed coherent atomic sources.  We show theoretically in this thesis that although it is possible to produce a continuous atom laser using an atomic reservoir, the flux achieved in the geometry considered is insufficient to compete with pulsed coherent atomic sources for precision measurement.  The results for the pumping process using an optical reservoir are more promising.  Although condensed sources were used as the source for this process, a detailed comparison of the theoretical calculations and experimental results indicate that the detrimental reabsorption processes are suppressed.  This suggests that it may be possible to use higher-flux thermal atomic sources to replenish the lasing mode of an atom laser with this pumping process.