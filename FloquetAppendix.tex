\chapter{Elementary excitations of temporally periodic Hamiltonians}
\label{FloquetAppendix}
\graphicspath{{Figures/FloquetAppendix/}{Figures/Common/}}

The theory of elementary excitations in unstable Bose-Einstein Condensates has been considered before \citep{Leonhardt:2003}. In this appendix, the methods discussed in \citep{Leonhardt:2003} will be applied to the case of temporally periodic but spatially homogeneous and isotropic Hamiltonians. One such Hamiltonian is obtained by making a perturbative expansion of the Hamiltonian \eqref{Peaks:InitialHamiltonian} considered in \sectionref{Peaks:PerturbativeApproach} about the time-dependent (but periodic) mean-field.

Consider a general quadratic Hamiltonian $\hat{H}(t)$ of period $T$ in terms of the operators $\hat{\phi}_i(\bm{x}, t)$ which obey the usual equal-time bosonic commutation relations. As the Hamiltonian $\hat{H}(t)$ is homogenous by assumption, the operator equations of motion will take their simplest form in a Fourier basis. In this basis the equations of motion for the operators $\hat{\phi}_i(\bm{k}, t)$ can be written in matrix form\footnote{FIXME: May need to add the argument that the Fourier transform must be of this form. It is simply a momentum conservation argument.},
\begin{subequations}
    \label{FloquetAppendix:MatrixOperatorEvolution}
    \begin{align}
        i \hbar \frac{\partial }{\partial t}\hat{\bm{\Upsilon}}(\bm{k}, t) &= \mathcal{H}(\bm{k}, t) \hat{\bm{\Upsilon}}(\bm{k}, t),\\
        \hat{\bm{\Upsilon}}(\bm{k}, t) &= 
        \begin{pmatrix}
            \hat{\phi}_1^{\phantom{\dagger}}(\bm{k}, t) &
            \hat{\phi}_1^\dagger(-\bm{k}, t) &
            \hat{\phi}_2^{\phantom{\dagger}}(\bm{k}, t) &
            \hat{\phi}_2^\dagger(-\bm{k}, t) &
            \dots
        \end{pmatrix}^\text{T},
    \end{align}
\end{subequations}
where the matrix $\mathcal{H}(\bm{k}, t)$ obeys
\begin{align}
        \mathcal{H}(\bm{k}, t+T) &= \mathcal{H}(\bm{k}, t),\\
        \mathcal{H}(\bm{k}, t) &= \mathcal{H}(-\bm{k}, t), \label{FloquetAppendix:HReflectionSymmetry}
\end{align}
where the last equality holds because $\hat{H}(t)$ is isotropic.

If the Hamiltonian $\hat{H}(t)$ were not time-dependent, $\mathcal{H}(\bm{k}, t)$ could be diagonalised to find the (potentially complex) eigenvalues $\Omega_j(\bm{k})$ and corresponding operators $\hat{Q}_j(\bm{k}, t)$ which would evolve as
\begin{align}
    \label{FloquetAppendix:ContinouousTimeEigenoperators}
    i \hbar \frac{\partial }{\partial t}\hat{Q}_j(\bm{k}, t) &= \hbar \Omega_j(\bm{k}) \hat{Q}_j(\bm{k}, t),
\end{align}
where the $\hat{Q}_j(\bm{k}, t)$ need not obey boson commutation relations. The real parts of the eigenvalues of $\mathcal{H}(\bm{k}, t)$ would give the excitation spectrum of the Hamiltonian $\hat{H}(t)$ with non-zero imaginary components giving the growth rates of the corresponding unstable mode.

In the case of a periodic matrix $\mathcal{H}(\bm{k}, t)$ it is instead the monodromy matrix $\mathcal{M}(\bm{k})$ (see \sectionref{Peaks:FloquetsTheorem}) that we wish to diagonalise. The monodromy matrix $\mathcal{M}(\bm{k})$ satisfies
\begin{align}
    \label{FloquetAppendix:MonodromyMatrix}
    \hat{\bm{\Upsilon}}(\bm{k}, nT) &= \mathcal{M}(\bm{k})^n \hat{\bm{\Upsilon}}(\bm{k}, 0),
\end{align}
where $n$ is a positive integer. In place of \eqref{FloquetAppendix:ContinouousTimeEigenoperators} we seek the operators $\hat{Q}_j(\bm{k}, t)$ that obey
\begin{align}
    \label{FloquetAppendix:QOperatorEvolution}
    \hat{Q}_j(\bm{k}, T) &= \lambda_j(\bm{k}) \hat{Q}_j(\bm{k}, 0),
\end{align}
where the $\hat{Q}_j(\bm{k}, t)$ are defined by
\begin{align}
    \label{FloquetAppendix:QOperatorDefinition}
    \hat{Q}_j(\bm{k}, t) &= \bm{c}_j^\dagger(\bm{k}) \hat{\bm{\Upsilon}}(\bm{k}, t),
\end{align}
for some vectors $\bm{c}_j(\bm{k})$, where $\bm{c}_j^\dagger(\bm{k})$ denotes the conjugate transpose. 

Using definitions \eqref{FloquetAppendix:MonodromyMatrix}--\eqref{FloquetAppendix:QOperatorDefinition}, it follows that the $\lambda_j(\bm{k})$ and $\bm{c}_j^\dagger(\bm{k})$ are respectively the eigenvalues and left eigenvectors of $\mathcal{M}(\bm{k})$,
\begin{align}
    \hat{Q}_j(\bm{k}, T) &= \bm{c}_j^\dagger(\bm{k}) \hat{\bm{\Upsilon}}(\bm{k}, T) = \bm{c}_j^\dagger(\bm{k}) \mathcal{M}(\bm{k}) \hat{\bm{\Upsilon}}(\bm{k}, 0) \\
    \hat{Q}_j(\bm{k}, T) &= \lambda_j(\bm{k}) \hat{Q}_j(\bm{k}, 0) = \lambda_j(\bm{k}) \bm{c}_j^\dagger \hat{\bm{\Upsilon}}(\bm{k}, 0)\\
    \implies \bm{c}_j^\dagger(\bm{k}) \mathcal{M}(\bm{k}) \hat{\bm{\Upsilon}}(\bm{k}, 0) &= \lambda_j(\bm{k}) \bm{c}_j^\dagger \hat{\bm{\Upsilon}}(\bm{k}, 0) \label{FloquetAppendix:LeftEigenvectorWithOperator}\\
    \implies \bm{c}_j^\dagger(\bm{k}) \mathcal{M}(\bm{k}) &= \lambda_j(\bm{k}) \bm{c}_j^\dagger(\bm{k}), \label{FloquetAppendix:LeftEigenvector}
\end{align}
where \eqref{FloquetAppendix:LeftEigenvector} follows from \eqref{FloquetAppendix:LeftEigenvectorWithOperator} as the components of $\hat{\bm{\Upsilon}}(\bm{k}, 0)$ are linearly independent operators.

The operators $\hat{Q}_j(\bm{k}, t)$ are not necessarily bosonic annihilation or creation operators. To determine the conditions under which they are, we consider their Hermitian conjugates $\hat{Q}_j^\dagger(\bm{k}, t)$. As every operator in $\hat{\bm{\Upsilon}}(-\bm{k}, t)$ is the Hermitian conjugate of an operator in $\hat{\bm{\Upsilon}}(\bm{k}, t)$, the $\hat{Q}_j^\dagger(\bm{k}, t)$ can be written as
\begin{align}
    \label{FloquetAppendix:QDaggerDefinition}
    \hat{Q}_j^\dagger(\bm{k}, t) &= \bm{d}_j^\dagger(\bm{k}) \hat{\bm{\Upsilon}}(-\bm{k}, t)
\end{align}
for some vectors $\bm{d}_j(\bm{k})$. It follows from \eqref{FloquetAppendix:QOperatorEvolution} that the $\hat{Q}_j^\dagger(\bm{k}, t)$ will obey
\begin{align}
    \label{FloquetAppendix:QDaggerEvolution}
    \hat{Q}_j^\dagger(\bm{k}, T) &= \lambda_j^*(\bm{k}) \hat{Q}_j^\dagger(\bm{k}, 0).
\end{align}

The commutators of the $\hat{Q}_j^{(\dagger)}(\bm{k}, t)$ will be constant as they are constant linear combinations of the $\hat{\phi}^{(\dagger)}(\pm\bm{k}, t)$, the commutators of which are themselves constant. Using this requirement gives
\begin{align}
    \left[ \hat{Q}_i^{\phantom{\dagger}}(\bm{k}, T),\, \hat{Q}_j^\dagger(\bm{k}, T) \right] &= \left[ \lambda_i(\bm{k}) \hat{Q}_i^{\phantom{\dagger}}(\bm{k}, 0),\, \lambda_j^*(\bm{k}) \hat{Q}_j^\dagger(\bm{k}, 0)\right]\\
        &= \lambda_i(\bm{k}) \lambda_j^*(\bm{k}) \left[ \hat{Q}_i^{\phantom{\dagger}}(\bm{k}, 0),\, \hat{Q}_j^\dagger(\bm{k}, 0)\right].
        \label{FloquetAppendix:InvariantCommutator}
\end{align}
For \eqref{FloquetAppendix:InvariantCommutator} to be true either $\lambda_i(\bm{k}) \lambda_j^*(\bm{k}) = 1$ or the two operators commute. Specifically, for $\hat{Q}_i(\bm{k}, t)$ to be an annihilation or creation operator it is required that ${\lambda_i^*(\bm{k})}^{-1} = \lambda_i(\bm{k})$. In terms of the Floquet exponents (see \sectionref{Peaks:FloquetsTheorem}) $\displaystyle \xi_i(\bm{k}) = \frac{1}{T} \ln \lambda_i(\bm{k})$, this requirement becomes $\xi_i(\bm{k}) = i \omega_i(\bm{k})$. Hence it is only for purely imaginary Floquet exponents that the eigenvalues of $\mathcal{M}(\bm{k})$ correspond to bosonic annihilation or creation operators. Note that in the degenerate case in which $\mathcal{H}(\bm{k}, t)$ is time-independent the Floquet exponents $\xi(\bm{k})$ are related to the eigenvalues $\Omega(\bm{k})$ of $\mathcal{H}(\bm{k}, t)$ by $\displaystyle \xi(\bm{k}) = -\frac{i}{\hbar} \Omega(\bm{k})$. Hence the imaginary component of the Floquet exponents are related to the excitation spectrum and non-zero real components are related to the existence of instabilities.

Generally the Floquet exponents $\xi_i$ may have a non-zero real component. In this case the $\hat{Q}_i(\bm{k}, t)$ will not be bosonic annihilation or creation operators, although such operators can be constructed from linear combinations of the $\hat{Q}_i(\pm\bm{k}, t)$. Before constructing such operators, we first consider the restrictions on the possible eigenvalues of $\mathcal{M}(\bm{k})$.

First it is noted that if $\lambda_i(\bm{k})$ is an eigenvalue of $\mathcal{M}(\bm{k})$, then from \eqref{FloquetAppendix:QDaggerDefinition}--\eqref{FloquetAppendix:QDaggerEvolution} $\lambda_i^*(\bm{k})$ must be an eigenvalue of $\mathcal{M}(-\bm{k})$. However, from the reflection symmetry of $\mathcal{H}(\bm{k})$ defined by \eqref{FloquetAppendix:HReflectionSymmetry} we have that $\mathcal{M}(-\bm{k}) = \mathcal{M}(\bm{k})$ and hence $\lambda_i^*(\bm{k})$ must also be an eigenvalue of $\mathcal{M}(\bm{k})$.

Secondly, not all of the operators $\hat{Q}_i^{(\dagger)}(\bm{k}, t)$ can commute. As the $\hat{Q}_i^{(\dagger)}(\bm{k}, t)$ form a complete basis over the same space as the $\hat{\phi}_i^{(\dagger)}(\bm{k}, t)$ which themselves do not all commute, for every operator $\hat{Q}_i(\bm{k}, t)$ there must be at least one other operator with which it does not commute. From \eqref{FloquetAppendix:InvariantCommutator} then follows the requirement that if $\lambda_i(\bm{k})$ is an eigenvalue,   ${\lambda_i^*(\bm{k})}^{-1}$ must also be an eigenvalue.

Combining these two requirements gives a consistency condition for the eigenvalues of $\mathcal{M}(\bm{k})$: if $\lambda$ is an eigenvalue, $\lambda^*$, $\lambda^{-1}$, and ${\lambda^*}^{-1}$ must all be eigenvalues. These conditions can be met using 1, 2 or 4 distinct eigenvalues of $\mathcal{M}(\bm{k})$.

For the degenerate case in which all of $\lambda$, $\lambda^*$, $\lambda^{-1}$, and ${\lambda^*}^{-1}$ are equal, the eigenvalue $\lambda = \pm 1$. The corresponding Floquet exponent is $\xi = 0$ or $\xi = i \pi\nu_0$ where $\displaystyle \nu_0 = T^{-1}$. This is not an interesting case and does not occur in \figureref{Peaks:CondensateEigenvalues}.

There are two ways that two distinct eigenvalues can be used to satisfy the consistency condition for the eigenvalues. The first possibility is that $\lambda = {\lambda^*}^{-1}$ (with $\lambda^*$ being the second eigenvalue). In this case the Floquet exponents are $\xi = \pm i \omega$. In this case, the operators corresponding to the eigenvalues are bosonic annihilation or creation operators as shown above. The second possibility is that $\lambda = \lambda^*$ (with $\lambda^{-1}$ being the second eigenvalue). In this case the Floquet exponents are $\xi = \pm \gamma$ or $\xi = \pm \gamma + i \pi \nu_0$. This situation is seen in \figureref{Peaks:CondensateEigenvalues} around $k \approx \unit[1.5 \times 10^6]{m}^{-1}$ and $k \approx \unit[2.25 \times 10^6]{m}^{-1}$.

The final possibility is that four distinct eigenvalues are used to satisfy the consistency condition. In this case the four eigenvalues $\lambda$, $\lambda^*$, $\lambda^{-1}$, and ${\lambda^*}^{-1}$ are different and the corresponding Floquet exponents are $\xi = \pm \gamma + i \omega$ and $\xi' = \pm \gamma - i\omega$. It is this situation that occurs in \figureref{Peaks:CondensateEigenvalues} around $k \approx \unit[0.75 \times 10^6]{m}^{-1}$.

In summary, the Floquet exponents with nonzero real parts come in pairs $\xi(\bm{k}) = \pm \gamma(\bm{k}) + i \omega(\bm{k})$. From the operators corresponding to these pairs of exponents bosonic annihilation and creation operators can be constructed.

Consider the eigenvalues $\displaystyle \lambda = e^{r + i \phi}$ and $\displaystyle \lambda' = e^{-r + i\phi}$, and the corresponding operators $\hat{Q}(\bm{k}, t)$ and $\hat{Q}'(\bm{k}, t)$. From these operators we define the following two operators which will respectively be shown to be bosonic annihilation and creation operators,
\begin{align}
    \hat{\Lambda}_1(\bm{k}, t) &= \frac{1}{\sqrt{2}} \left( \hat{Q}(\bm{k}, t) + \hat{Q}'(\bm{k}, t)\right),\\
    \hat{\Lambda}_2(\bm{k}, t) &= \frac{1}{\sqrt{2}} \left( \hat{Q}(\bm{k}, t) - \hat{Q}'(\bm{k}, t)\right).
\end{align}
As $\lambda \lambda'^* = 1$, $\hat{Q}(\bm{k}, t)$ and $\hat{Q}'^\dagger(\bm{k}, t)$ will not commute. By appropriate rescaling of the operators, we can define their commutator to be
\begin{align}
    \left[ \hat{Q}(\bm{k}, t),\, \hat{Q}'^\dagger(\bm{k}, t) \right] &= 1.
\end{align}
This choice then defines the value of the other nonzero commutator,
\begin{align}
    \left[ \hat{Q}'(\bm{k}, t),\, \hat{Q}^\dagger(\bm{k}, t) \right] &= 1.
\end{align}

From these two commutators it can then be shown that $\hat{\Lambda}_1(\bm{k}, t)$ obeys the commutation relations appropriate for an annihilation operator, while $\hat{\Lambda}_2(\bm{k}, t)$ obeys the commutation relations for a creation operator. For example,
\begin{align}
    \left[\hat{\Lambda}_1^{\phantom{\dagger}}(\bm{k}, t),\, \hat{\Lambda}_1^\dagger(\bm{k}, t) \right] &= \frac{1}{2} \left[ \hat{Q}(\bm{k}, t) + \hat{Q}'(\bm{k}, t),\, \hat{Q}^\dagger(\bm{k}, t) + \hat{Q}'^\dagger(\bm{k}, t)\right] = 1.
\end{align}

Defining $\hat{\Lambda}_1'(-\bm{k}, t) = \hat{\Lambda}_2^\dagger(\bm{k}, t)$, the evolution of the operators $\hat{\Lambda}_1(\bm{k}, t)$ and $\hat{\Lambda}_1'(\bm{k}, t)$ can now be determined. $\hat{\Lambda}_1(\bm{k}, t)$ evolves as
\begin{align}
    \hat{\Lambda}_1(\bm{k}, nT) &= \frac{1}{\sqrt{2}} \left(\hat{Q}(\bm{k}, nT) + \hat{Q}'(\bm{k}, nT) \right) \\
        &= \frac{1}{\sqrt{2}} \left( e^{nr + i n\phi} \hat{Q}(\bm{k}, 0) + e^{-nr + i n \phi}\hat{Q}'(\bm{k}, 0)\right) \\
        \begin{split}
            &=  e^{i n \phi} \bigg[ \frac{1}{2}\left( e^{nr} - e^{-nr}\right)\frac{1}{\sqrt{2}}\left(\hat{Q}(\bm{k}, 0) - \hat{Q}'(\bm{k}, 0) \right) \\
            &\relphantom{=e^{i n \phi}\bigg[} + \frac{1}{2} \left( e^{nr} + e^{-nr}\right)\frac{1}{\sqrt{2}} \left( \hat{Q}(\bm{k}, 0) + \hat{Q}'(\bm{k}, 0)\right)\bigg]
        \end{split}\\
        &= e^{i n \omega T} \left( \sinh(n\gamma T) \hat{\Lambda}_1'^\dagger(-\bm{k}, 0) + \cosh(n \gamma T) \hat{\Lambda}_1(\bm{k}, 0)\right), \label{FloquetAppendix:LambdaOperatorEvolution}
\end{align}
where $n$ is a positive integer. Similarly, $\hat{\Lambda}_1'(\bm{k}, t)$ can be shown to evolve as
\begin{align}
    \hat{\Lambda}_1'(\bm{k}, nT) &= e^{-i n \omega T} \left( \sinh(n \gamma T) \hat{\Lambda}_1^\dagger(-\bm{k}, 0) + \cosh(n \gamma T) \hat{\Lambda}_1'(\bm{k}, 0)\right). \label{FloquetAppendix:LambdaPrimeOperatorEvolution}
\end{align}
The evolution represented by \eqref{FloquetAppendix:LambdaOperatorEvolution} and \eqref{FloquetAppendix:LambdaPrimeOperatorEvolution} is the same as that for the non-degenerate parametric oscillator \citep{WallsMilburn} which is known to generate EPR entanglement between the $\hat{\Lambda}_1(\bm{k}, t)$ and $\hat{\Lambda}_1'(-\bm{k}, t)$ modes.

Finally, if the $\hat{Q}(\bm{k}, t)$ and $\hat{Q}'(\bm{k}, t)$ operators correspond to only two distinct eigenvalues (i.e. $\lambda = \lambda^*$), then by the uniqueness of the eigenvectors $\hat{\Lambda}_1'(\bm{k}, t) = \hat{\Lambda}_1(\bm{k}, t)$. For this case $\displaystyle e^{i n\omega T} = \pm 1$, making \eqref{FloquetAppendix:LambdaOperatorEvolution} and \eqref{FloquetAppendix:LambdaPrimeOperatorEvolution} consistent.

