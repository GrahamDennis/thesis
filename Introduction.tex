\chapter{Introduction}
\label{Introduction}
\graphicspath{{Figures/Introduction/}{Figures/Common/}}

\section{Atom lasers}
\label{Introduction:AtomLaser}

\section{Metastable helium}
\label{Introduction:Helium}

\begin{table}
    \centering
    \begin{tabular}{ccc}
    \toprule
    Parameter & Value\\
    \midrule
    Quasimolecule $S=0$ scattering length & $a_{S=0}=\unit[9.46]{nm}$\\
    Quasimolecule $S=2$ scattering length & $a_{S=2}=\unit[7.51]{nm}$\\
    Penning ionisation rate  & $\Kunpol = \unit[7.7\times 10^{-17}]{m\textsuperscript{3}}$ & \citep{Stas:2006kx}\\
    \bottomrule
    \end{tabular}
    \caption{\label{BackgroundTheory:He*Parameters} Relevant parameters of metastable helium.}
\end{table}



\section{Pumping and (atom) lasers}
\label{Introduction:Pumping}

FIXME: This content to go in the introduction.

A brief overview of what gives a laser its properties. Refer to Wiseman's paper~\citep{Wiseman:1997ba} and compare the optical and atom lasers. Alternatives for the irreversibility / source are discussed in \chapterref{OpticalPumping,KineticTheory}.

There are two fundamental reasons making the pumping of an atom laser harder than pumping an optical laser is that the dispersion relation for potential reservoirs is not flat by comparison to the dispersion relation for atoms.  For an optical laser in the homogeneous broadening limit, \emph{all} atoms in the sample can contribute to gain.  Fundamentally this is because independent of what momentum the atom might have, it can still be stimulated to emit a photon into the lasing mode as the decay rates of the excited and/or ground atomic states are greater than any possible energy detuning in this process.  The second fundamental issue affecting atom lasers is that atom number is conserved.  It is therefore inescapable that there will be atoms in a (potentially) non-condensed source mode in the vicinity of the lasing mode.  Unless a Feschbach resonance is used to set the scattering-length of atomic interactions to zero, the atoms in the source mode will disrupt the phase stability of the lasing mode.

In \chapterref{OpticalPumping}, we attempt to solve the first problem by making the momentum distribution of source atoms sufficiently narrow that it can be guaranteed that every atom will be momentum-resonant with the pumping process at some point.  In \chapterref{KineticTheory}, we use atomic modes as the reservoir.  Although their dispersion relation cannot be flat with respect to that of the atoms in the source mode, it is at least far more comparable than that of light (which is used in \chapterref{OpticalPumping}).

Ideally we should pump an atom laser using something with a flat dispersion relation.  One possibility that comes to mind is the phonon.  However it is necessary that the reservoir is essentially a pure vacuum.  Phonons produced in a condensate cannot escape the system unless the system is in contact with other atoms.  If these other atoms are thermal, than it will have its own phonons which will propagate into the condensate.  If the condensate is not in physical contact with anything, then the phonons cannot leave the system: there is no evaporation process for phonons.

Stolen from \citep{Ballagh:2000oq}:
Despite the parallels with optical laser theory, the fundamental differences between photons and atoms remain important. They arise principally because atoms have rest mass, and because they interact with each other. Unlike photons, atoms cannot be created or destroyed, rather they are transferred into the laser mode from some other mode. The interactions produce phase dynamics, which degrades the coherence of the atom laser, and self repulsion which spreads the output beam and limits the possible focussing. Even when these interactions are neglected, atoms have dispersive propagation in vacuum.
