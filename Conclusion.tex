\chapter{Conclusion}
\label{Conclusion}
\graphicspath{{Figures/Conclusion/}{Figures/Common/}}

Is it possible to produce a continuous coherent atomic source with sufficient flux to be viable for use in atomic interferometry?  That is the question that this thesis has sought to contribute to.  For the specific realisations of the pumping mechanisms considered in this thesis, the answer is no; however the answer to the question more generally is less clear.  

For the optical pumping mechanism considered in \chapterref{OpticalPumping}, the achievable flux of the atom laser is limited by the rate at which condensates can be produced to replace the source condensate as it is depleted.  Even if all the atoms were transferred to the lasing condensate with perfect efficiency, the continuous atom laser produced could never exceed the average flux with which the source condensates can be produced.  The optical pumping experiment described in that chapter is therefore not a useful source for atom interferometry in its present form.  However, there is some evidence that the harmful reabsorption processes in that experiment may be suppressed due to a quantum-mechanical interference effect.  A better theoretical understanding of this process would determine whether or not the source condensate could be replaced with a thermal source.  If possible, this would greatly increase the potential flux of the produced atom laser.

The evaporatively-driven pumping mechanism considered in \chapterref{KineticTheory}, is supplied by a thermal source and therefore it could be possible for it to provide a condensed source of atoms at a similar flux to the thermal source.  However, the actual flux is necessarily lower as the evaporation process removes atoms from the system as part of the pumping mechanism.  One of the aims of \chapterref{KineticTheory} was to determine just how much lower the flux of the condensed source would be.  It was found that the largest flux achievable from an available thermal atomic source was $\sim\unit[10^5]{s\textsuperscript{-1}}$, an average flux that is also achievable by producing independent condensates.

While neither of the proposed pumped atom lasers would be competitive with thermal sources for atom interferometry, pumping could improve other atom laser experiments.  In particular, experiments in which squeezing or entanglement of atomic beams is being produced and measured.  In these experiments, the atomic beam will be condensed as thermal sources 

In these experiments, the atomic beam will be condensed as a thermal source is inadequate for 

 necessarily be condensed


In these situations, the beams themselves will 


In these situations, a thermal source is inadequate, and a condensed source is necessary.  For these experiments, a pumped atom laser could potentially offer the narrow linewidth and continuous operation necessary to increase the sensitivity of measurements of squeezing and entanglement.  


Now we move on to a discussion of Peaks.

Despite this bleak outlook, this thesis has some more positive conclusions.  

Peaks:

In this chapter, an unusual process in an atom laser was investigated.  The process involved the direct conversion of mean-field energy to the kinetic energy of unstable modes in the condensate.  This process was shown to generate entanglement in certain regimes, although in the experiment discussed it is unlikely that this entanglement remains in the outcoupled atom laser, and certainly not in any useful form.  These problems are however not fundamental.  A differently-designed experiment could overcome some of these problems to potentially produce entangled atom lasers.

The below is a positive conclusion:

One possibility worthwhile investigating would be to use a highly elongated two-state condensate in which both states experience the same trapping potential.  This could be achieved for example through the use of an optical dipole trap.  As the two states of the condensate would experience the same trapping potential, the two components of the dynamical instabilities would propagate together, avoiding the problem of one of the components of the entangled modes leaving the condensate without the other.  Further, due to the high aspect ratio, the instabilities would propagate solely along the axial dimension.  To extract the entangled beams, one would need to turn off the optical dipole trap after a given interaction time.  The atoms would then expand ballistically, with the entangled beams spatially separating from the main condensate due to their large momentum in the axial direction.  This experiment may then enable the production of highly-directional entangled atom lasers.  Further theoretical investigation would be necessary to determine if such an experiment were feasible.


Optical pumping:

There are three parts to the optical pumping process underlying the pumping experiments discussed in this chapter:
\begin{enumerate}[(i)]
    \item the delivery of atoms in an appropriate mode to the lasing condensate;
    \item the transfer of those atoms into the lasing condensate; and
    \item the propagation of the emitted photons within the lasing condensate.
\end{enumerate}
This chapter has aimed to understand part (ii) of this process.  Ideally we would also like to understand (iii) as it appears that there is some very interesting physics occurring there (at least for the $2 \hbar k$ momentum-transfer process), however there is strong experimental evidence that physics beyond the mean-field are significant in this part of the process and more detailed theoretical modelling will be necessary to accurately describe this part of the process.  By contrast process (i) is a detail determined by how one chooses to get the atoms to the lasing condensate with an appropriate mode.  While it can be argued that this process is well understood in the context of the pulsed pumping experiment, significant approximations were used for the case of the continuous pumping experiment.  While the details of the transfer of atoms into an appropriate source mode for the pumping mechanism are of course important, they are not of fundamental importance to the pumping mechanism.

One of the questions relevant to the continuous pumping experiment that we investigated theoretically was whether it was the $0 \hbar k$ or $2 \hbar k$ momentum-transfer process operating in the experiment.  While theoretically we were unable to find a regime in which the $2 \hbar k$ momentum-transfer process delivered efficient pumping, there is experimental evidence (see \citep{Doring:2009} for details) that the outcoupling position in the continuous pumping experiment was closer to the centre of the source condensate, not near the edge of the condensate as necessary for the operation of the $0 \hbar k$ momentum-transfer process.  As discussed in the previous section, it may be due to the neglect of the reabsorption process that the $2 \hbar k$ momentum-transfer process was not found to operate efficiently.  This does not discount the result that the $0 \hbar k$ momentum-transfer process \emph{can} operate.  As the photons are emitted at the edge of the condensate and propagate away reabsorption does not significantly affect this process.  It is somewhat surprising that there is a regime in which a resonant optical pumping process can be operated in which the resonant photons do not get the chance to interact significantly with the lasing condensate.

The comparison of theoretical and experimental results suggests that physics beyond the mean field are significant in the $2 \hbar k$ momentum-transfer process, which is certainly operating at least in the pulsed pumping experiment.  Further investigation of the counter-intuitive possibility that reabsorption may decrease heating resulting from spontaneous emission is certainly warranted.  The next step in this process would be to theoretically model and solve the full 3D atom--light system treating spontaneous emission and reabsorption fully.  This will be a computationally intensive process and too prohibitive to apply to the continuous pumping experiment.  However it could reasonably be applied to the pulsed pumping experiment where the optical degrees of freedom only need to be included during the short time for which the pumping light is applied.  The fall of the atomic pulses under gravity may be treated separately using standard techniques.

The complication resulting from the reabsorption of resonant photons does not arise when the applied optical pumping light is significantly detuned.  In this limit it has been shown that in a simple atom laser model pumping efficiencies of about 50\% are achievable.  It is interesting to note that the relative propagation direction of the emitted light and atom laser can have a significant effect on the efficiency of the pumping process in the detuned limit.  Although practical operation in this model was limited to the $0 \hbar k$ momentum-transfer process, both processes may be feasible in a model closer to the continuous pumping experiment in which the effects of gravity are included.  In such a model the atom laser will only be momentum-resonant with the lasing condensate for a short amount of time, reducing the possibility that multiple Rabi oscillations will limit the efficiency (refer to \figureref{OpticalPumping:TravellingBeamFarDetunedResults}).  This would be a straightforward extension of the present work.


The below is a positive conclusion:

Contrary to what has been expected previously, this chapter has demonstrated that optical pumping of an atom laser is feasible without making the condensate so narrow as to make it essentially transparent in at least one dimension.  The next step in the investigation of this pumping process must be the theoretical consideration of the potential positive contribution of reabsorption.  When this is better understood, consideration can then be given to the integration of the pumping mechanism with a method for replacing the source condensates to produce a continuously pumped atom laser.


Kinetic Theory:

The purpose of this chapter has been to investigate the feasibility of producing a continuously pumped atom laser driven by  collisions with a cloud of thermal atoms.  The method has been to investigate the best-case scenario in which the replenishment process introduces no heating to the trapped thermal component beyond that due to bringing the replenishing atoms into contact with the thermal cloud.  With these caveats in mind, the results are promising: using an existing experimental source~\citep{Muller:2007} it appears possible to produce steady-state condensates with large atom number ($\sim 5\times 10^5$ atoms) using the scheme presented in \sectionref{KineticTheory:Scheme}.  Should the atomic flux of this source be increased by an order of magnitude, the condensate number produced by this scheme could be pushed to $5\times 10^6$ atoms.  

Ultimately it is not the size of the condensate that we are interested in, but the flux of the atom laser produced and its coherence length.  The former goal will certainly be improved by larger equilibrium condensate sizes, however the atom laser itself will be useless unless its coherence length is sufficiently larger than that of cold thermal sources to offset its necessarily lower atom flux.  The model used in this chapter cannot answer the latter question.  To investigate this further it will be necessary to include the full multimode behaviour of the Raman outcoupler and the fluctuations of the thermal cloud.  This can be achieved through using a more general kinetic theory such as the `ZNG theory'~\citep{Zaremba:1999,Proukakis:2008} or the Stochastic Projected Gross-Pitaevskii equation (SPGPE)~\citep{Blakie:2008a}.


I need specific conclusions that can be drawn from the thesis, and a brief discussion of future work.  I should make some positive noises about thermal pumping.  I should make some noises about just how many orders of magnitude in flux need to be gained to reach the same level as the cold atomic sources currently used in atom interferometers.  After all, atom interferometry was the stated motivation at the start of this thesis.

On optical pumping: the result is far less clear.  Optical pumping is certainly hard, due to the dispersion relation problems discussed in the Introduction, but there are some interesting behaviours occurring in the pumping systems that warrant further investigation.  While perhaps less satisfying, the results are intriguing.

On entangled beams: the system is not sufficiently clean to be used as a source of entangled atomic beams.  Other problems discussed with Rob would seem to prevent the system being used for quasi-continuous sources either.  Pulsed mode appears to be the best we can do.  We have, however, demonstrated good agreement between theory and experiment, and that must count for something.

Transverse profile:  Large interaction strengths are likely to be a problem.  Interactions are, however, necessary in the evaporation to BEC.  This can be controlled through the use of a Feschbach resonance.  Other work has demonstrated that the atom laser can be guided to a high degree.

