\chapter{Calculational tools}
\label{ToolsAppendix}
\graphicspath{{Figures/ToolsAppendix/}{Figures/Common/}}

% This is an appendix in which we \emph{briefly} (hah!) summarise the features and methods of \texttt{xpdeint} with particular attention paid to those features used in this thesis. Of course appropriate acknowledgement must be made of \texttt{XMDS}. Standing on the shoulders of giants and all that.

Although not theoretical physics of itself, a not insignificant part of any theoretical physics work is spent in the use of computational tools. It is unsurprising therefore that just as experimental PhD candidates will spend a significant portion of their time building and designing the apparatus for experiments, so too will a theorist spend time developing the tools that they need in the pursuit of their work. In this appendix I describe the primary tool that I have developed in the course of my PhD. %I would like to think that the work described here might be used beyond the term of my PhD and more widely than by myself. more widely find wider application beyond wider application

While none of the techniques described in this appendix are new, the utility of \texttt{xpdeint} is the ease with which such a wide range of algorithms and methods can be applied. While the generality cannot be compared with such tools as MATLAB and Mathematica, the advantage is in the focussed design and speed advantages (for large problems).

\begin{itemize}
    \item Spectral method
    \item IP operator
    \item Various basis functions
    \item Coupled ODE-PDE systems
    \item Gaussian quadrature
    \item Stochastic integration
    \item Distributed simulations
    \item Cross-propagation
\end{itemize}