\chapter{Optical pumping of an atom laser}
\label{OpticalPumping}
\graphicspath{{Figures/OpticalPumping/}{Figures/Common/}}

The results presented in this chapter have been published in \citet{Robins:2008,Doring:2009}.

\section{Motivation}

The development of the continuous-wave optical laser was a significant advance over the first pulsed ruby laser. The continuous-wave optical laser opened up many applications. The atom laser is a very promising source for both precision measurement and fundamental physics.

The replenishment process can be divided into two critical components: a delivery system for filling an atomic reservoir with ultracold atoms and a pumping mechanism for irreversibly and continuously transferring atoms from the reservoir to the laser mode.

The technical requirements on both parts of the replenishment system are stringent. Nonetheless, recent experiments have demonstrated that a delivery system for atoms is feasible and possible. \citet{Chikkatur:2002qa} showed that Bose-condensed atoms could be periodically transported over large distances using a moving optical dipole trap. Further experiments with transport, based on interference of two counter-propagating lasers, have shown that dipole trapping techniques could be extended to provide continuous delivery of atoms \citep{Schmid:2006}. Magnetic guiding systems for ultracold atoms may also provide a path to future delivery systems \citep{Lahaye:2004,Greiner:2001,Greiner:2007}.

The realisation of the pumping mechanism for a continuous atom laser has proved more problematic. There are four critical requirements that are difficult to satisfy experimentally. First, the atoms should enter the laser mode continuously and coherently, that is, with the phase and amplitude of the lasing condensate. Thus, atoms must make a transition that is Bose-stimulated by the atomic lasing mode. The second requirement is that the pumping process is irreversible. It requires coupling to a reservoir. There are two reservoirs available, the empty modes of the electromagnetic field accessible via a transition from an excited atomic state, or the empty modes of the atomic field accessible via evaporation. For a high-coherence atom laser, the lasing mode must be a pure condensate with a significantly smaller thermal fraction, making evaporation-induced pumping a difficult possibility for the production for a highly coherent continuous atom laser. The third requirement is that the pumping system must be compatible with a continuous replenishment mechanism. This suggests strongly that there be a physical separation between the source and the lasing condensates. A physical separation with a stimulated transition between the source and the lasing mode isolates the lasing mode from phase kicks and heating that would result either as a necessary consequence of the replenishment system (for example in the replenishment system demonstrated by \citeauthor{Chikkatur:2002qa} where condensates are merged \citep{Chikkatur:2002qa}) or as a consequence of an imperfect delivery system. Finally, the fourth condition on a pumping system is that it should be possible to continuously output-couple atoms from the laser mode into a beam, while the pumping mechanism is operating.

The two possibilities for reservoirs to supply the irreversibility necessary for the proper operation of a pumped atom laser (Bose-Einstein condensate) are considered in this chapter and the next. In this chapter, pumping an atom laser using interactions mediated by light is considered. In this case the reservoir providing the irreversibility are the vacuum electromagnetic modes into which light is scattered by the pumping process. \chapterref{KineticTheory} considers the alternate possibility of using direct atom--atom interactions to mediate the pumping process with evaporation to make the process irreversible.
