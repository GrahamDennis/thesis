\chapter{Derivations and calculations}
\label{MethodsAppendix}
\graphicspath{{Figures/MethodsAppendix/}{Figures/Common/}}

%Some content goes here.

% Perhaps this appendix might be called `technical details' or the like. And then we could cover things to keep an eye out for when doing simulations. Things like running off the edge of k-space causing phantom reflections, imaginary time algorithms. I could imagine quite a summary of techniques being possible.
\section{Proof of the periodicity of the nonlinear optical Bloch equations}
\label{MethodsAppendix:OpticalBlochPeriodicityProof}
In this section a proof is given of the periodicity of the solutions to the nonlinear optical Bloch equations considered in \sectionref{Peaks:MeanFieldPeriodicity} which described the dynamics of the mean field of a two-level homogeneous condensate in the case that the scattering lengths for the two levels are not equal. We restate the nonlinear optical Bloch equations \eqref{Peaks:OpticalBlochEquations} here for convenience,
\begin{subequations}
    \label{MethodsAppendix:OpticalBlochEquations}
    \begin{align}
        \frac{d}{dt}\rho_{10} &= -i\frac{g}{2} (1-w)\rho_{10} + i \Omega w,\\
        \frac{d }{dt}w &= -4 \Omega \Im\{\rho_{10}\},
    \end{align}
\end{subequations}
where $\rho_{10}$ is the off-diagonal element of the density matrix, and $w = \rho_{11} - \rho_{00}$.

These equations describe motion on the surface of a sphere, known as the Bloch sphere (see \figureref{Peaks:BlochSphere}), with each point on the sphere corresponding to a different physical state.  A generalisation of the Poincaré-Bendixson theorem that applies to compact, connected, two-dimensional orientable manifolds \citep{Schwartz:1963} (such as the surface of the sphere) states that every trajectory either approaches one or more fixed points, or approaches a periodic orbit.  The possibility of a space-filling trajectory is precluded as the surface of a sphere is not homeomorphic to a torus (see \citep{Schwartz:1963}).  Periodic trajectories and fixed points trivially approach themselves.

We next aim to show that there are no limit cycles in this system, i.e.\ there exist no trajectories that approach periodic orbits which are not themselves periodic.

We first write the equations of motion for the system in the usual spherical polar coordinates as
\begin{subequations}
    \label{MethodsAppendix:EvolutionOfAngles}
    \begin{align}
        \frac{d \theta}{dt} &= 2 \Omega \sin\phi, \label{MethodsAppendix:EvolutionOfAngles:Theta}\\
        \frac{d \phi}{dt} &= 2 \Omega \cos\phi \cot\theta - \frac{1}{2} g (1-\cos\theta), \label{MethodsAppendix:EvolutionOfAngles:Phi}
    \end{align}
\end{subequations}
where we note that these equations are of the form
\begin{align}
    \dot{\vect{r}} &= \frac{2}{\hbar} \unitvec{r} \times \nabla E, \label{MethodsAppendix:ConservativeEvolution}
\end{align}
where $E$ is the energy per particle previously defined in \eqref{Peaks:OpticalBlochEnergy}, and an inverted hat is used to denote unit vectors.

As the evolution has the form \eqref{MethodsAppendix:ConservativeEvolution}, a given point moves in the direction perpendicular to the gradient of $E$ at a rate proportional to the magnitude of the gradient, and hence the energy $E$ is conserved along any given trajectory.

Assuming a limit cycle exists, there will exist a trajectory approaching the limit cycle which is not the limit cycle. For each point in the limit cycle $\vect{x}$ there exists an infinite sequence of points $\vect{s}_i$ that approach $\vect{x}$ which are the intersection of a curve which intersects no trajectory tangentially (a \emph{transversal}, see \citep{Schwartz:1963}) and the trajectory approaching the limit cycle.  As the energy $E$ is continuous, we have $\displaystyle \lim_{i\rightarrow\infty}E(\vect{s}_i) = E(\vect{x})$, and therefore the energy of the approaching trajectory must be the same as the energy of the limit cycle.  The derivative of $E$ in the direction of the transversal is zero as $\displaystyle \lim_{i\rightarrow\infty} \frac{E(\vect{s}_i) - E(\vect{x})}{\abs{\vect{s}_i - \vect{x}}} = 0$.  It is already known that the derivative of the energy is zero parallel to the limit cycle as energy is conserved along any trajectory. As the direction of the transversal and the limit cycle at $\vect{x}$ are linearly independent (a transversal intersects no trajectory tangentially) and the manifold is two-dimensional, the gradient of $E$ at $\vect{x}$ must necessarily be zero.  From \eqref{MethodsAppendix:ConservativeEvolution} this implies that $\vect{x}$ is a fixed point, which is a contradiction with $\vect{x}$ being part of a limit cycle.  Our original assumption that a limit cycle exists is therefore false.

It now remains to be shown that no trajectories approach fixed points (with the exception of the fixed points themselves).  With the possibility of limit cycles excluded, by the generalised Poincaré-Bendixson theorem this will show that all remaining trajectories are periodic.

To analyse the stability of the fixed points, it is first necessary to determine their location.  It immediately follows from \eqref{MethodsAppendix:EvolutionOfAngles:Theta} that fixed points will satisfy $\sin\phi = 0$, i.e.\ fixed points are restricted to the great circle in the $x$--$z$ plane.  To parameterise this circle by a single coordinate, we alter the usual ranges of the spherical polar coordinates such that $\phi \in \left[-\frac{\pi}{2}, \frac{\pi}{2}\right)$ and $\theta \in [-\pi, \pi)$.  With this definition $\cos\phi = 1$ over the entire circle, instead of $\cos\phi = \pm 1$ over opposite halves.

Using the tan-half substitution $s = \tan\frac{1}{2}\theta$, it can be shown from \eqref{MethodsAppendix:EvolutionOfAngles:Phi} that all fixed points satisfy
\begin{align}
    g s^3 &= \Omega (1 - s^4). \label{MethodsAppendix:FixedPointsEquation}
\end{align}

The stability of \eqref{MethodsAppendix:EvolutionOfAngles} about the fixed points may be investigated to demonstrate that there are no trajectories that approach the fixed points (with the exception of the fixed points themselves).  Linearising \eqref{MethodsAppendix:EvolutionOfAngles} about the fixed points yields
\begin{align}
    \frac{d}{dt} \delta \theta &= 2 \Omega \delta \phi,\\
    \frac{d}{dt} \delta \phi &= - \frac{\Omega}{2 s^2}(s^4 + 3) \delta \theta,
\end{align}
where $\delta \theta$ and $\delta\phi$ are the deviations from the fixed point and the identity \eqref{MethodsAppendix:FixedPointsEquation} has been used to simplify the result.  The eigenvalues of this linear system of differential equations are
\begin{align}
    \lambda &= \pm\sqrt{- \frac{\Omega^2}{s^2}(s^4 + 3)},
\end{align}
which are pure imaginary as both $\Omega$ and $s$ are real.  This implies that there exist no trajectories near any fixed points of the system that asymptotically approach the fixed point.  It may then be concluded that with the exception of the fixed points themselves, all trajectories of the system \eqref{MethodsAppendix:OpticalBlochEquations} are periodic, and Floquet's theorem may be applied in \sectionref{Peaks:ExperimentEigenvalues}.

\section{Example calculation of the momentum density flux}
\label{MethodsAppendix:MomentumDensityFluxExampleCalculation}

\begin{figure}
    \centering
    \includegraphics[width=8cm]{AbsorbingBoundaryLayerScattering}
    \caption{\label{MethodsAppendix:AbsorbingBoundaryLayerScattering} An incident wave of wavenumber $k_0$ incident on an absorbing boundary layer. The region of interest is the part of the computational domain in which the absorbing potential $V_I(x)$ is zero. An auxiliary layer is added outside of the absorbing boundary layer as a model for a number of artificial boundary conditions (see main text).}
\end{figure}

As a demonstration of the efficacy of the method described in \sectionref{Peaks:AbsorbingBoundaryTricks} for determining the rate of loss of momentum density from a region of space, we consider a wave of wavenumber $k_0$ incident from the left on an imperfect absorbing boundary layer in a 1D computational domain (see \figureref{MethodsAppendix:AbsorbingBoundaryLayerScattering}), and compare $\Phi(k, t)$ (refer to \eqref{Peaks:MomentumDensityFlux}) to the result expected in the case of a perfect absorbing boundary layer of $\displaystyle \frac{\hbar k_0}{M}\delta(k - k_0)$. In this example $s$-wave scattering will be neglected. As the computational domain in this example is effectively infinite, the wavefunction used in the evaluation of \eqref{Peaks:MomentumDensityFlux} will be restricted to be nonzero only in the absorbing boundary layer to demonstrate the finite momentum resolution obtainable from this method due to the (except in this example) finite extent of the computational domain.

As a model for a number of different artificial boundary conditions we consider there to be \emph{no} artificial boundary condition at the edge of the computational domain, and instead it to be surrounded by an `auxiliary layer' in which there is a negative imaginary potential the reflection of that in the absorbing boundary layer. The negative imaginary potential is then symmetric about the edge of the computational domain. In the case of periodic boundary conditions, the auxiliary layer will correspond to the absorbing boundary layer on the other side of the computational domain in which it is assumed that the reflected negative imaginary potential is used. In the case of Dirichlet or Neumann boundary conditions in which respectively the wavefunction or its derivative is set to zero on the boundary, the auxiliary layer corresponds to the absorbing boundary layer reflected. In either of these latter two cases, the wavefunction for the actual artificial boundary conditions will be a linear combination of the wavefunction \emph{without} the artificial boundary conditions in the absorbing boundary layer and in the auxiliary layer. Specifically, in the case of Dirichlet boundary conditions in which the wavefunction is set to zero on the boundary, the wavefunction in the presence of the artificial boundary condition $\psi_\text{abc}(\vect{x})$ will be given by $\psi_\text{abc}(\vect{x}) = \psi(\vect{x}) - \psi(-\vect{x})$ where $\psi(\vect{x})$ is the wavefunction in the absence of the artificial boundary condition, and $x=0$ corresponds to the edge of the computational domain.

To calculate $\Phi(k, t)$ from \eqref{Peaks:MomentumDensityFlux} it is necessary to know the solution $\psi(x)$ to the time-independent Schrödinger equation subject to the boundary conditions that there is an incident wave from the left with wavenumber $k_0$ and no incident wave from the right. Given two linearly independent solutions to the Schrödinger equation in the doubled absorbing boundary layer (the absorbing boundary layer / auxiliary layer region), $\psi(x)$ can be found by applying these boundary conditions. It now remains to obtain two linearly independent solutions to the time-independent Schrödinger equation within the doubled absorbing boundary layer.

Solutions to the time-independent Schrödinger equation within the doubled absorbing boundary layer can be obtained with relative ease in one dimension as it is simply an ordinary differential equation,
\begin{align}
    -\frac{\hbar^2}{2M}\frac{d^2 \psi}{dx^2} + V(x) \psi(x) &= E(k_0) \psi(x) = \frac{\hbar^2 k_0^2}{2M} \psi(x),
\end{align}
which is equivalent to
\begin{align}
    \frac{d^2 \psi}{dx^2} &= \frac{2 M}{\hbar^2} V(x) \psi(x) - k_0^2 \psi(x).
    \label{MethodsAppendix:1DTimeIndependentSchrodingerEquation}
\end{align}
Two linearly independent (but not necessarily orthogonal) solutions $\phi_1(x)$, $\phi_2(x)$ to \eqref{MethodsAppendix:1DTimeIndependentSchrodingerEquation} can be found by simply choosing two linearly independent initial conditions and numerically propagating the solutions through the potential $V(x) = -i V_I(x)$. The solution $\psi(x) = c_1 \phi_1(x) + c_2 \phi_2(x)$ can then be found by requiring continuity of the wavefunction and its derivative at the left and right edges of the doubled absorbing boundary layer,
\begin{subequations}
    \label{MethodsAppendix:1DBoundaryConditions}
    \begin{align}
        e^{i k x} + \alpha_R e^{-i k x} \Big|_\text{left} &=  c_1 \phi_1(x) + c_2 \phi_2(x) \Big|_\text{left}, \\
        \frac{d}{dx}\big( e^{i k x} + r e^{-i k x}\big) \Big|_\text{left} &=  \frac{d}{dx} \big( c_1 \phi_1(x) + c_2 \phi_2(x) \big) \Big|_\text{left},\\
        \alpha_T e^{i k x} \Big|_\text{right} &= c_1 \phi_1(x) + c_2 \phi_2(x) \Big|_\text{right}, \\
        \frac{d}{dx} \big( \alpha_T e^{i k x} \big) \Big|_\text{right} &= \frac{d}{dx} \big( c_1 \phi_1(x) + c_2 \phi_2(x) \big) \Big|_\text{right},
    \end{align}
\end{subequations}
where $\alpha_R$ and $\alpha_T$ are the reflected and transmitted amplitudes respectively. 

As a by-product of solving \eqref{MethodsAppendix:1DBoundaryConditions} for $\psi(x)$, the reflection and transmission fractions $R=\abs{\alpha_R}^2$ and $T=\abs{\alpha_T}^2$ respectively can be obtained, giving a quantitative description of the effectiveness of a given absorbing boundary layer. For reflecting artificial boundary conditions, the reflection coefficient is $R'= \abs{\alpha_R \mp \alpha_T}^2$ respectively for Dirichlet and Neumann boundary conditions, and as $\alpha_R$ and $\alpha_T$ can not both be significant simultaneously for any absorbing boundary layer that is effective over some finite range of wavenumbers, the approximation $R' \approx \max(R, T)$ can be used.  Hence $R$ and $T$ are useful measures of the effectiveness of an absorbing boundary layer independent of the artificial boundary conditions used.

\begin{figure}
    \centering
    \includegraphics[width=14cm]{AbsorbingBoundaryLayerEffectiveness}
    \caption{\label{MethodsAppendix:AbsorbingBoundaryLayerEffectiveness} The reflection $R$ and transmission $T$ coefficients from a typical absorbing boundary layer as a function of wavenumber. The potential used was $\displaystyle V_I(x) = \hbar \omega \cos^2\left( \frac{\pi x}{2 \Delta x} \right)$ where $\omega = \unit[5 \times 10^4]{rad.s\textsuperscript{-1}}$ and $\Delta x = \unit[5]{\micro m}$ is the size of the absorbing boundary layer. Also marked on this figure are the approximate lower and upper bounds of the effectiveness of the absorbing boundary layer as given by \eqref{BackgroundTheory:AbsorbingBoundaryKEffectiveRange}.}
\end{figure}

In \figureref{MethodsAppendix:AbsorbingBoundaryLayerEffectiveness} the reflected and transmitted fractions $R$ and $T$ are plotted as a function of the incident wavenumber and a comparison is made to the approximate range of validity of the absorbing boundary as given by \eqref{BackgroundTheory:AbsorbingBoundaryKEffectiveRange}. 


With $\psi(x)$ determined, the steady state momentum density flux $\Phi_{k_0}(k)$ can be obtained for a given incident wavenumber $k_0$. This distribution is plotted in \figureref{MethodsAppendix:PhiAccuracy} for the same absorbing boundary used in \figureref{MethodsAppendix:PhiAccuracy}. As any real absorbing boundary layer will have finite extent, the resolution of $\Phi_{k_0}(k)$ will be limited by $\Delta k = \pi/\Delta x$, where $\Delta x$ is the width of the absorbing boundary layer. This finite resolution will prevent $\Phi_{k_0}(k)$ reproducing the exact result in the limit of a perfect absorbing boundary layer of $\displaystyle \frac{\hbar k_0}{M}\delta(k - k_0)$. As a measure of the accuracy of $\Phi_{k_0}(k)$, its integral over a range of a few $\Delta k$ should be compared to the exact answer. To this aim we define
\begin{align}
    \eta(k_0) &= \left(\frac{\hbar k_0}{M}\right)^{-1} \int_{k_0-5\Delta k}^{k_0 + 5\Delta k} dk\, \Phi_{k_0}(k),
\end{align}
where $\eta(k_0)$ is plotted in \figureref{MethodsAppendix:AbsorbingBoundaryLayerEffectiveness}. As expected, $\eta \approx 1$ over the same range of incident wavenumbers for which the absorbing boundary layer is effective.

\begin{figure}
    \centering
    \includegraphics[width=8cm]{PhiAccuracy}
    \caption{\label{MethodsAppendix:PhiAccuracy} The momentum flux density $\Phi_{k_0}(k)$ leaving the region of interest in \figureref{MethodsAppendix:AbsorbingBoundaryLayerScattering} as a function of the incident wavenumber $k_0$. As expected, $\Phi_{k_0}(k)$ is sharply peaked around $k=k_0$. The resolution of $\Phi_{k_0}(k)$, $\Delta k$ is indicated by the width of the $k=k_0$ line.}
\end{figure}

\section[Solving the Quantum Kinetic Theory model]{Solving the Quantum Kinetic Theory model of \chapterref{KineticTheory}}
\label{MethodsAppendix:KineticTheory}

One of the difficulties involved in solving the kinetic model of \chapterref{KineticTheory} is that the energy range that the problem is defined over changes in time.  The maximum energy is simply the energy of the evaporative cut-off $\varepsilon_\text{cut}$, while the minimum energy is the chemical potential of the condensate $\mu(t)$.  A discretisation of the energy dimension over the range $[0, \varepsilon_\text{cut}]$ will suffer from problems accurately representing the lower-end of the distribution where the minimum energy of the thermal atoms is varying.  An alternative is write the problem in terms of a shifted energy coordinate $\overline{\varepsilon} \equiv \varepsilon - \mu(t)$ so that the minimum energy of the system is now fixed.  Of course the same problem now exists at the upper end of the energy range where the maximum energy $\overline{\varepsilon}_\text{max} = \varepsilon_\text{cut} - \mu(t)$ is now time-dependent.  However, at equilibrium there will be significantly fewer thermal atoms at the evaporation cut-off than there will be near the condensate [this is illustrated in \figureref{KineticTheory:EnergyDistributionFunctionEvolution}(a)].  This choice will then result in smaller numerical errors than the alternative.

Written in terms of the shifted energy variable $\overline{\varepsilon}$, the contribution due to the replenishment is
\begin{align}
    \left. \frac{\partial\big(\overline{\rho}(\overline{\varepsilon}, t) \overline{g}(\overline{\varepsilon}, t)\big)}{\partial t}\right|_\text{replenishment} &= \Gamma \overline{\rho}_0(\overline{\varepsilon}) \overline{g}_T(\overline{\varepsilon}),
    \label{MethodsAppendix:KineticTheory:ReplenishmentProcess}
\end{align}
where a bar over a function is used to indicate that it is defined in terms of the shifted energy coordinate.  The original form of this term is given by \eqref{KineticTheory:ReplenishmentProcess}.

\subsection{Density of states}
\label{MethodsAppendix:QKTDensityOfStates}

Although the evolution equations for the kinetic model \eqref{KineticTheory:EvolutionEquations} are written in terms of the product of the density of states $\rho(\varepsilon, t)$ and the energy distribution function $g(\varepsilon, t)$, it will be necessary to separately determine the energy distribution function to evaluate the collisional contributions given in the following section. To extract the energy distribution function it is necessary to have an explicit expression for the density of states for the thermal cloud. This density of states is not simply the same as that for a harmonic trap as the thermal modes will experience a mean-field repulsion due to the condensate mode. The effective potential experienced by the thermal atoms is
\begin{align}
    V_\text{eff}(\vect{r}, t) &= V_\text{trap}(\vect{r}) + 2 g n_c(\vect{r}, t), \label{MethodsAppendix:QKTEffectivePotential}
\end{align}
where $V_\text{trap}(\vect{r})$ is the potential due to the magnetic trap, $g = 4\pi \hbar^2 a/m$, $a$ is the \emph{s}-wave scattering length and $n_c(\vect{r}, t)$ is the condensate density which was assumed to follow a Thomas-Fermi distribution in \chapterref{KineticTheory}.  Note that the `2' in the above expression is the full Hartree-Fock mean field experienced by the thermal atoms (see \eqref{Peaks:ThermalParticleEnergySpectrum}, or~\citep[Chapter 8]{PethickSmith} for further details) which is twice the mean-field repulsion experienced by condensate atoms.  This is essentially due to the thermal atoms being distinguishable from the condensate atoms, while the condensate atoms are indistinguishable from one another.

The density of states in the presence of the effective potential \eqref{MethodsAppendix:QKTEffectivePotential} is given by
\begin{align}
    \rho(\varepsilon, t) &= \int \frac{d \vect{r}\, d \vect{p}}{(2\pi \hbar)^3} \,\delta\left(\varepsilon - V_\text{eff}(\vect{r}, t) - \vect{p}^2/2 m\right).
    \label{MethodsAppendix:DensityOfStatesDefinition}
\end{align}
The integrals are performed in~\citep{Bijlsma:2000} giving the following result in terms of the shifted energy coordinate (Eqs.~49 and 50 in~\citep{Bijlsma:2000})
\begin{align}
    \overline{\rho}(\overline{\varepsilon}, t) &= \frac{2}{\pi \hbar \overline{\omega}} \left[I_-(\overline{\varepsilon}) + I_+(\overline{\varepsilon})\right],
\end{align}
where the functions $I_\pm(\overline{\varepsilon})$ are
\begin{align}
    I_-(\overline{\varepsilon}) &= \left.\frac{u_-^3 x}{4} - \frac{a_- u_- x}{8} - \frac{a_-^2}{8}\ln(x + u_-)\right|_{x=\sqrt{\max\{0, -a_-\}}}^{x=\sqrt{2\mu/\hbar \overline{\omega}}}, \label{MethodsAppendix:QKTIMinus}\\
    I_+(\overline{\varepsilon}) &= \left.- \frac{u_+^3 x}{4} + \frac{a_+ u_+ x}{8} + \frac{a_+^2}{8} \arcsin\left(\frac{x}{\sqrt{a_+}}\right)\right|_{x=\sqrt{2\mu/\hbar\overline{\omega}}}^{x=\sqrt{a_+}}, \label{MethodsAppendix:QKTIPlus}
\end{align}
with $a_\pm = 2(\overline{\varepsilon}\pm \mu)/\hbar\overline{\omega}$, and $u_\pm = \sqrt{a_\pm \mp x^2}$. Note that there is a minor typo in~\citet{Bijlsma:2000}, the lower limit of $I_-(\overline{\varepsilon})$ is given as $x=\sqrt{\max\{0, a_-\}}$, while it should read $x=\sqrt{\max\{0, -a_-\}}$ as in \eqref{MethodsAppendix:QKTIMinus}.

\subsection{Collision and energy-redistribution in Quantum Kinetic Theory}
\label{MethodsAppendix:QKTOtherTerms}

The forms of the collision and energy-redistribution terms of the kinetic model described in \chapterref{KineticTheory} were omitted there for sake of clarity as their derivation was not part of the work presented there.  A full derivation of these terms is given in~\citep{Bijlsma:2000,Proukakis:2008}.

The contribution due to thermal--thermal collisions is given in Eq.~26 of~\citep{Bijlsma:2000} and has the form
\begin{align}
    \begin{split}
        \left. \frac{\partial\big(\overline{\rho}(\overline{\varepsilon}_1, t) \overline{g}(\overline{\varepsilon}_1, t)\big)}{\partial t}\right|_\text{thermal--thermal} &= \frac{m^3 g^2}{2 \pi^3 \hbar^7} \int d\overline{\varepsilon}_2 \int d\overline{\varepsilon}_3 \int d\overline{\varepsilon}_4 \,\overline{\rho}(\overline{\varepsilon}_\text{min}, t)\\
        &\relphantom{=} \times\delta(\overline{\varepsilon}_1 + \overline{\varepsilon}_2 - \overline{\varepsilon}_3 - \overline{\varepsilon}_4) \\
        &\relphantom{=} \times [ (1+\overline{g}_1) (1+\overline{g}_2) \overline{g}_3 \overline{g}_4 - \overline{g}_1 \overline{g}_2 (1+\overline{g}_3) (1+\overline{g}_4)],
    \end{split}
\end{align}
where $\overline{\varepsilon}_\text{min}$ is the minimum of the $\overline{\varepsilon}_i$, and $\overline{g}_i = \overline{g}(\overline{\varepsilon}_i, t)$. 

The contribution due to thermal--condensate collisions is given by Eq.~53 and Eq.~58--60 of~\citep{Bijlsma:2000} and has the form
\begin{align}
    \begin{split}
        \left. \frac{\partial\big(\overline{\rho}(\overline{\varepsilon}_1, t) \overline{g}(\overline{\varepsilon}_1, t)\big)}{\partial t}\right|_\text{thermal--condensate} &= \frac{m^3 g^2}{2 \pi^3 \hbar^7} \int d \overline{\varepsilon}_2 \int d\overline{\varepsilon}_3 \int d\overline{\varepsilon}_4 \,\delta(\overline{\varepsilon}_2 - \overline{\varepsilon}_3 - \overline{\varepsilon}_4)\\
        &\relphantom{=}\times\left[ \delta(\overline{\varepsilon}_1 - \overline{\varepsilon}_2) - \delta(\overline{\varepsilon}_1 - \overline{\varepsilon}_3) - \delta(\overline{\varepsilon}_1- \overline{\varepsilon}_4)\right]\\
        &\relphantom{=} \times \left[(1+\overline{g}_2)\overline{g}_3 \overline{g}_4 - \overline{g}_2 (1+\overline{g}_3)(1+\overline{g}_4) \right]\\
        &\relphantom{=}\times  \int_{\overline{U}_\text{eff}(\vect{r}, t) \leq \overline{U}_-} d \vect{r}\, n_c(\vect{r}, t),
    \end{split}
    \label{MethodsAppendix:QKTRhoGThermalCondensateEvolution}
\end{align}
where $\displaystyle \overline{U}_- = \frac{2}{3}\left[(\overline{\varepsilon}_3 + \overline{\varepsilon}_4)-\sqrt{\overline{\varepsilon}_3^2 - \overline{\varepsilon}_3 \overline{\varepsilon}_4 + \overline{\varepsilon}_4^2}\right]$, and $\overline{U}_\text{eff}(\vect{r}, t) = U_\text{eff}(\vect{r}, t) - \mu(t)$.
The corresponding contribution to the evolution of the condensate number is simply
\begin{align}
    \left. \frac{d N_0}{d t}\right|_\text{thermal--condensate} &= - \int d\overline{\varepsilon} \,\left. \frac{\partial\big(\overline{\rho}(\overline{\varepsilon}, t) \overline{g}(\overline{\varepsilon}, t)\big)}{\partial t}\right|_\text{thermal--condensate}.
    \label{MethodsAppendix:QKTNThermalCondensateEvolution}
\end{align}


Finally, the contribution due to energy redistribution is (Eqs.~32 and 52 in~\citep{Bijlsma:2000})
\begin{align}
    \left. \frac{\partial\big(\overline{\rho}(\overline{\varepsilon}_1, t) \overline{g}(\overline{\varepsilon}_1, t)\big)}{\partial t}\right|_\text{redistribution} &= - \frac{\partial \big( \overline{\rho}_\text{w} \overline{g}\big)}{\partial \overline{\varepsilon}},
    \label{MethodsAppendix:QKTRedistributionEvolution}
\end{align}
where $\overline{\rho}_\text{w}$ is the weighted density of states
\begin{align}
    \overline{\rho}_\text{w}(\overline{\varepsilon}) &= \frac{2}{\pi \hbar \overline{\omega}} \left[ I_-(\overline{\varepsilon}) - I_+(\overline{\varepsilon})\right] \frac{d \mu}{dt},
\end{align}
where the functions $I_\pm(\overline{\varepsilon})$ are given in \eqref{MethodsAppendix:QKTIMinus} and \eqref{MethodsAppendix:QKTIPlus}.


\subsection{Three-body loss in Quantum Kinetic Theory}
\label{MethodsAppendix:QKT3BodyLoss}

The dominant density-dependent loss process in Bose-Einstein condensates is three-body loss~\citep{Burt:1997fk,Soding:1999}. In \chapterref{KineticTheory} it was argued that three-body loss was an important process in the operation of the pumped atom laser scheme proposed there.  The following derivation of the three-body loss contribution to the kinetic model \eqref{KineticTheory:EvolutionEquations} was performed by \emph{Matthew Davis} and is presented for completeness.

Three-body loss (or three-body recombination) is the process in which three atoms collide forming a bound dimer with the third necessary to ensure both energy and momentum conservation.  The binding energy is sufficient to give the products of a three-body recombination process sufficient kinetic energy to rapidly escape the trap.  Three-body loss is then well-described by the master equation term
\begin{align}
    \left. \frac{d \hat{\rho}}{d t}\right|_\text{3-body loss} &= \frac{1}{3}L_3 \int d \vect{x} \,\mathcal{D} \left[ \hat{\Psi}^3(\vect{x}) \right] \hat{\rho},
    \label{MethodsAppendix:ThreeBodyLossMasterEquationTerm}
\end{align}
where $\mathcal{D}[\hat{c}]\hat{\rho} = \hat{c}\hat{\rho} \hat{c}^\dagger - \frac{1}{2}(\hat{c}^\dagger \hat{c}\hat{\rho} + \hat{\rho} \hat{c}^\dagger \hat{c})$ is the usual decoherence superoperator, and $L_3 = \unit[5.8\times 10^{-30}]{cm\textsuperscript{6}s\textsuperscript{-1}}$~\citep{Burt:1997fk} is the three-body recombination loss rate constant.  This equation, first derived by~\citep{Jack:2002} has the familiar form of a decoherence superoperator with the state undergoing loss as the argument (cf. \sectionref{PenningIonisationAppendix:MasterEquation}). %This is a reasonable approximation as three-body recombination couples the state $\hat{\Psi}^3$ to a state that is rapidly depopulated.

The loss rate of atoms from the system due to three-body loss is readily obtained from \eqref{MethodsAppendix:ThreeBodyLossMasterEquationTerm} as
\begin{align}
    \left.\frac{d N}{dt} \right|_\text{3-body loss} &=  \Tr\left\{\int d \vect{r}\,\hat{\Psi}^\dagger(\vect{r})\hat{\Psi}(\vect{r}) \left.\frac{d \hat{\rho}}{dt}\right|_\text{3-body loss} \right\} = - L_3 \int d \vect{r}\, \mean{\hat{\Psi}^\dagger(\vect{r})^3 \hat{\Psi}(\vect{r})^3}.
    \label{MethodsAppendix:ThreeBodyLossNumberLossRate}
\end{align}

To separate the contributions to \eqref{MethodsAppendix:ThreeBodyLossNumberLossRate} due to the thermal and condensed components, we use an approach similar to that of the Bogoliubov theory discussed in \sectionref{Peaks:ElementaryExcitations}.  We write the annihilation operator $\hat{\Psi}$ in terms of its mean value $\Psi \equiv \mean{\hat{\Psi}}$ and the fluctuation operator $\delta \hat{\Psi} \equiv \hat{\Psi} - \Psi$ and substitute this into \eqref{MethodsAppendix:ThreeBodyLossNumberLossRate}.  In contrast to \sectionref{Peaks:ElementaryExcitations} in which the zero-temperature limit was considered, the fluctuation operator defined here represents thermal fluctuations, which cannot be considered to be small.  Higher powers of $\delta\hat{\Psi}$ can therefore not be neglected.  However, thermal fluctuations have no well-defined phase relationship to one another or to the condensate.  Expectation values containing an unequal number of creation and annihilation fluctuation operators such as $\mean{\delta\hat{\Psi}\delta\hat{\Psi}}$ can therefore be assumed to be zero (cf. \eqref{Peaks:HamiltonianPowerSeriesQuadraticTerm} in which the $\delta\hat{\Psi}^\dagger \delta\hat{\Psi}^\dagger$ and $\delta\hat{\Psi}\delta\hat{\Psi}$ terms were retained as there is a well-defined phase relationship between the quasiparticles and the condensate).

Performing the substitution described, \eqref{MethodsAppendix:ThreeBodyLossNumberLossRate} becomes
\begin{align}
    \begin{split}
        \left.\frac{d N}{dt} \right|_\text{3-body loss} &=  -L_3 \int d \vect{r}\, \Big\{[n_c(\vect{r})]^3 + 9 [n_c(\vect{r})]^2 \mean{\delta\hat{\Psi}^\dagger(\vect{r}) \delta\hat{\Psi}(\vect{r})}\\
        &\relphantom{=-L_3 \int d \vect{r}\,\Big\{} + 9 n_c(\vect{r}) \mean{\delta\hat{\Psi}^\dagger(\vect{r})^2 \delta\hat{\Psi}(\vect{r})^2} + \mean{\delta\hat{\Psi}^\dagger(\vect{r})^3 \delta\hat{\Psi}(\vect{r})^3}\Big\},
    \end{split}
    \label{MethodsAppendix:ThreeBodyLossNumberLossRateInTermsOfFluctuationOperators}
\end{align}
where $n_c(\vect{r}) = \abs{\Psi(\vect{r})}^2$ is the condensate density.

The non-condensate density is given by $n_T(\vect{r}) = \mean{\delta\hat{\Psi}^\dagger(\vect{r}) \delta\hat{\Psi}(\vect{r})}$.  As thermal states are Gaussian, the higher-order expectation values in the previous expression may be simplified by the application of Wick's theorem~\citep{Wick:1950} giving
\begin{align}
    \mean{\delta\hat{\Psi}^\dagger(\vect{r})^2 \delta\hat{\Psi}(\vect{r})^2} &= 2 [n_T(\vect{r})]^2, \\
    \mean{\delta\hat{\Psi}^\dagger(\vect{r})^3 \delta\hat{\Psi}(\vect{r})^3} &= 6 [n_T(\vect{r})]^3.
\end{align}
Substituting these expressions back into \eqref{MethodsAppendix:ThreeBodyLossNumberLossRateInTermsOfFluctuationOperators} yields
\begin{align}
    \left.\frac{d N}{dt} \right|_\text{3-body loss} &=  -L_3 \int d \vect{r}\, [n_c(\vect{r})]^3 + 9 [n_c(\vect{r})]^2 n_T(\vect{r}) + 18 n_c(\vect{r}) [n_T(\vect{r})]^2 + 6 [n_T(\vect{r})]^3.
    \label{MethodsAppendix:ThreeBodyLossNumberLossRateInTermsOfDensities}
\end{align}

The evaluation of this loss rate requires the evaluation of the condensate and thermal densities.  The condensate density $n_c(\vect{r})$ is fully determined by the condensate occupation $N_0(t)$ within the Thomas-Fermi approximation that has already been made elsewhere in the derivation of the kinetic model.  The first term of \eqref{MethodsAppendix:ThreeBodyLossNumberLossRateInTermsOfDensities} only involves the condensate density and may be evaluated analytically
\begin{align}
    \frac{d N_0}{dt} &= - L_3 \frac{15^{4/5}}{168 \pi^2} \left(\frac{m \overline{\omega}}{\hbar \sqrt{a}} \right)^{12/5} N_0^{9/5}.
\end{align}
The remaining terms of \eqref{MethodsAppendix:ThreeBodyLossNumberLossRateInTermsOfDensities} require an expression for the thermal density $n_T(\vect{r})$, which can be obtained from the energy distribution function $g(\varepsilon)$ and the density of states $\rho(\varepsilon)$.

The total number of thermal atoms $N_T$ can be written as
\begin{align}
    N_T &= \int d\varepsilon\, \rho(\varepsilon) g(\varepsilon),
    \label{MethodsAppendix:NThermalDefinition}
\end{align}
where the density of states is defined by \eqref{MethodsAppendix:DensityOfStatesDefinition}.  Substituting this into \eqref{MethodsAppendix:NThermalDefinition} and rearranging the order of integrals gives
\begin{align}
    N_T &= \int d \vect{r}\, \int d\varepsilon\, \rho(\varepsilon, \vect{r}) g(\varepsilon),
    \label{MethodsAppendix:NThermalInTermsOfRhoER}
\end{align}
where we have defined
\begin{align}
    \rho(\varepsilon, \vect{r}) &= \int d \vect{p}\, \delta\left(\varepsilon - V_\text{eff}(\vect{r}, t) - \vect{p}^2/2m\right) = \frac{m^{3/2}}{\sqrt{2} \pi^2 \hbar^3} \sqrt{\varepsilon - V_\text{eff}(\vect{r})}.
\end{align}
The thermal density can be identified from \eqref{MethodsAppendix:NThermalInTermsOfRhoER}
\begin{align}
    n_T(\vect{r}) &= \int d\varepsilon\, \rho(\varepsilon, \vect{r}) g(\varepsilon).
    \label{MethodsAppendix:nThermalInTermsOfRhoER}
\end{align}

The remaining terms of \eqref{MethodsAppendix:ThreeBodyLossNumberLossRateInTermsOfDensities} can now be expressed in terms of the energy distribution function $g(\varepsilon)$ and the density of states $\rho(\varepsilon)$ by substituting \eqref{MethodsAppendix:nThermalInTermsOfRhoER} for one of the factors of $n_T(\vect{r})$ in each term
\begin{align}
    -L_3 \int d \vect{r}\, 9 [n_c(\vect{r})]^2 n_T(\vect{r}) &= - L_3 \int d\varepsilon\, g(\varepsilon) \int d \vect{r}\, 9 \rho(\varepsilon, \vect{r}) [n_c(\vect{r})]^2, \label{MethodsAppendix:Condensate2Thermal1}\\
    -L_3 \int d \vect{r}\, 18 n_c(\vect{r}) [n_T(\vect{r})]^2 &= - L_3 \int d\varepsilon\, g(\varepsilon) \int d \vect{r}\, 18 \rho(\varepsilon, \vect{r}) n_c(\vect{r}) n_T(\vect{r}), \label{MethodsAppendix:Condensate1Thermal2}\\
    -L_3 \int d \vect{r}\, 6 [n_T(\vect{r})]^3 &= - L_3 \int d\varepsilon\, g(\varepsilon) \int d \vect{r}\, 6 \rho(\varepsilon, \vect{r}) [n_T(\vect{r})]^2. \label{MethodsAppendix:Condensate0Thermal3}
\end{align}
From these expressions the rate of loss of atoms of energy $\varepsilon$ from the distribution can be identified
\begin{align}
    \left.\frac{\partial\big(\rho(\varepsilon) g(\varepsilon)\big)}{\partial t}\right|_\text{3-body loss} &= -L_3 \int d \vect{r}\, \rho(\varepsilon, \vect{r}) g(\varepsilon) \left\{ 3 [n_c(\vect{r})]^2 + 12 n_c(\vect{r}) n_T(\vect{r}) + 6 [n_T(\vect{r})]^2 \right\},
    \label{MethodsAppendix:QKT3BodyLossDistributionEvolution}
\end{align}
where the contributions due to the terms involving only one or two thermal atoms have been multiplied by $1/3$ and $2/3$ respectively to share appropriately the total loss.  The corresponding term for the condensate number evolution is
\begin{align}
    \begin{split}
        \left. \frac{d N_0}{dt} \right|_\text{3-body loss} &= - L_3 \frac{15^{4/5}}{168 \pi^2} \left(\frac{m \overline{\omega}}{\hbar \sqrt{a}} \right)^{12/5} N_0^{9/5}\\
        &\relphantom{=} - L_3 \int d\varepsilon \int d \vect{r}\, \rho(\varepsilon, \vect{r}) g(\varepsilon) \left\{6 [n_c(\vect{r})]^2 + 6 n_c(\vect{r}) n_T(\vect{r}) \right\},
    \end{split}
    \label{MethodsAppendix:QKT3BodyLossCondensateEvolution}
\end{align}
where the contributions due to the terms involving only one or two condensate atoms have been multiplied by $1/3$ and $2/3$ respectively.

% The value of the three-body loss rate constant used in \chapterref{KineticTheory} was that obtained by~\citet{Burt:1997fk} of $L_3 = \unit[5.8\times 10^{-30}]{cm\textsuperscript{6}s\textsuperscript{-1}}$ (denoted $K_3^{c}$ in that work).



%%%%%%%%%%%%%%%%%%%%%%%%%%%%%%%%%%%%%%%%%%%%%%%%%%%%%%%%%%%%%%%%%%%%%%%%%%%%%%%%%%%%%%%%%%%%%%%%%%%%%%%%%%%%%%

% \section{Convolutions}
% Unrelated to thesis but important: When doing convolutions you must truncate the range of the interaction potential to prevent interacting with the `copies' of the density that you do want to interact with. This Fourier transform \emph{must} be calculated analytically. The range of the grid must also be such that the density is nonzero within a rectangular prism with side lengths $\frac{1}{2}$ that of the computational domain. This is to prevent the left part of the density interacting with the right part due to the wrap-around~\citep{Ronen:2006}.
% \begin{align}
%     \mathcal{F}(f * g) &= \sqrt{2 \pi} \mathcal{F}(f) \mathcal{F}(g)
% \end{align}
