\chapter{Conclusion}
\label{Conclusion}
\graphicspath{{Figures/Conclusion/}{Figures/Common/}}

Atom lasers show extraordinary promise.  Using atoms instead of photons for interferometric measurements potentially increases the sensitivity by up to 10 orders of magnitude.  Although much of this gain is offset due to the much lower fluxes of available atomic sources, current atom interferometers are competitive or world-leading in the precision measurement of accelerations and rotations.  Coherent atomic sources (i.e.\ atom lasers) have fluxes even lower than the available thermal atomic sources that are used in current atom interferometers.  Coherent sources do, however, have two potentially significant advantages over thermal sources for atom interferometry.

The first is their narrower linewidth, and therefore velocity spread.  This enables more efficient operation of large momentum-transfer processes (which are highly velocity selective).  Large momentum-transfer processes are important for increasing the interaction time of the atoms, and therefore the phase shift to be measured.  Also, for larger interaction times the expansion of the atomic source increases signal loss \citep{Dimopoulos:2007uq}, an effect which is reduced by the use of a condensed source.

The second advantage of coherent sources is their potential for producing squeezed and entangled atomic sources.  In the context of quantum optics, squeezing has been used to reduce the shot noise limit (the standard quantum limit $\propto 1/\sqrt{N}$ where $N$ is the number of particles observed) in precision measurement by about a factor of 10 \citep{Vahlbruch:2008,Mehmet:2010}.  This is equivalent to an increase in the flux of the source of a factor of 100.  Entanglement has the possibility of even more dramatic improvements, however it relies on the creation of very fragile entangled sources.  The best that has been achieved in the context of quantum optics are entangled sources containing less than 10 photons \citep{Leibfried:2004}.  In the context of atom optics, squeezed atomic sources offer the possibility of increasing measurement sensitivity, while entangled sources are of greater interest for fundamental tests of quantum mechanics.

Atom lasers, as a technology, are still in their infancy.  The first pulsed atom lasers were demonstrated in 1997, and since then researchers have extracted them from condensates quasi-continuously, guided them, split them, and probed condensates with them.  They are an atomic analogue of the photon laser, except in one respect: they are not truly continuous.  When the condensate runs out of atoms, the atom laser stops.  Present-day atom lasers are more analogous to a leaky cavity than a laser in this respect.  

Creating a truly continuous coherent source for atoms is tricker than for photons.  Perhaps the largest challenge is that atom number is conserved.  It is not simply possible by adding more energy to the system to create additional atoms (at least not without adding a \emph{lot} of energy) like it is for photons.  A source of atoms is necessary to produce a truly continuous atom laser, and a mechanism is needed to replenish the condensate using this source, and the replenishment process must operate without significantly disturbing the coherence properties of the condensate.  It is this replenishment or pumping process that we have investigated theoretically in this thesis.

There are two choices for the reservoir providing the irreversibility for a pumping process for an atom laser: the empty modes of an optical field, or the empty modes of an atomic field.  A pumping process of each form has been considered in each of \chapterref{OpticalPumping,KineticTheory}.

In \chapterref{OpticalPumping}, a pumping mechanism in which the system is driven optically was considered.  In this process, the emission of photons makes the pumping process irreversible.  This process has the advantage that atoms are not necessarily lost in the evaporation process, potentially making the pumping process highly efficient.  In an experimental realisation of this scheme using a condensate as the source, a 35\% efficiency for this transfer process was observed.  The flux of this particular realisation, however, would be limited by the rate at which the source condensates can be produced.  The interesting property of this system, however, is that following a detailed comparison of the theoretical and experimental results, there is some intriguing evidence that the harmful reabsorption processes in the experiment may be suppressed due to a quantum-mechanical interference effect.  A better theoretical understanding of this process could enable the source condensate to be replaced with a thermal source.  If possible, this would greatly increase the potential flux of the produced atom laser, creating a competitive coherent atomic source for atom interferometry.

A second, evaporatively-driven pumping process was considered in \chapterref{KineticTheory}.  The removal of atoms from the system due to evaporation makes this process irreversible.  This pumping process is supplied by a thermal source, and could therefore have a higher flux than any pumping process supplied by condensates.  However, the actual flux of the produced coherent source is necessarily lower as the evaporation process removes atoms from the system as part of the pumping mechanism.  One of the aims of \chapterref{KineticTheory} was to determine just how much lower the flux of the condensed source would be.  It was found that the largest flux achievable from an available thermal atomic source was $\sim\unit[10^5]{s\textsuperscript{-1}}$ operating at an efficiency of just 0.05\%.  As this was a `best-case scenario'-type calculation, the fact that a similar, if not larger flux can be achieved by producing independent condensates essentially rules out this type of pumped atom laser as a viable alternative to thermal sources for atom interferometry.  There is also some experimental evidence that the proposed geometry is not appropriate for the operation of a pumped atom laser \citep{Robins:2008}.  This does not rule out other geometries for evaporatively-driven pumping mechanisms such as that proposed by \citet{Mandonnet:2000lr} in which a guided thermal atomic beam is evaporatively cooled \emph{transversely} as it propagates.  By decreasing the evaporation cut-off as the beam propagates, it may be cooled to degeneracy forming a continuous beam of coherent atoms.  This proposal has been investigated experimentally \citep{Cren:2002rt,Roos:2003vn,Roos:2003,Lahaye:2004,Vogels:2004,Lahaye:2005uq}, in which it was found that transverse thermalisation occurred too slowly for evaporation to proceed efficiently.  There has been a recent reinvigoration of experimental effort in this direction \citep{Hempel:2008,Traxler:2009,Traxler:2010}.

While neither of the proposed pumped atom lasers in their present forms would be competitive with thermal sources for atom interferometry, pumping could improve other atom laser experiments.  In particular, experiments in which squeezing or entanglement of atomic beams is being produced and measured.  In these experiments, the atomic beam will be condensed as it would be more difficult to detect squeezing or entanglement with a thermal source.  For these experiments, a pumped atom laser could potentially offer a narrower linewidth, increased flux, and continuous operation, all of which would increase the sensitivity of measurements of atomic squeezing and entanglement.

\chapterref{Peaks} discussed one such process that leads to the formation of entangled atomic beams.  In the experiment discussed in that chapter, multimode effects limit the potential for this entanglement to be observed directly.  It is likely that only number-difference squeezing and correlations in atom detection events could be observed in this experiment.  A different experiment was proposed in which these multimode effects do not occur, potentially enabling the entanglement of the produced atomic beams to be measured.  This chapter has also demonstrated the excellent agreement that is achievable between theory and experiment for BEC systems.  This is possible due to extraordinary degree of control over external noise sources that is possible in experiments, and the effectiveness of the theoretical techniques available for modelling them.

\parasep

This thesis documents the work of a theorist working, for the most part, in close collaboration with experimentalists.  This is particularly evident in \chapterref{Peaks} and \chapterref{OpticalPumping}, but it is also true for other work not documented here.  I have attempted to resolve the minor mysteries that sometimes arise in experimental physics by working closely with the experimentalists as the discovery is made, causes confusion, and for the most part, is resolved.

% Now for a strong finish.


% Contrary to what has been expected previously, this chapter has demonstrated that optical pumping of an atom laser is feasible without making the condensate so narrow as to make it essentially transparent in at least one dimension.  The next step in the investigation of this pumping process must be the theoretical consideration of the potential positive contribution of reabsorption.  When this is better understood, consideration can then be given to the integration of the pumping mechanism with a method for replacing the source condensates to produce a continuously pumped atom laser.



