\chapter*{Abstract}
%\addcontentsline{toc}{chapter}{Abstract}
\graphicspath{{Figures/Common/}}


250-500 words.


Measurement is our fundamental tool for learning about the world around us.  It is from observing trends in measurements that we develop theories that enable us to predict future behaviour, and it is against measurements that we determine the validity of these theories.  Increases in the precision of our measurements are fundamental to understanding.

Atom interferometry is a new method for performing precision measurements developed in 1991.  It uses the matter-wave nature of atoms to perform interferometry experiments analogous to those performed with photons.  However, in contrast to optical interferometry, which uses coherent sources of photons, atom interferometry uses thermal sources, in part due to the unavailability of high-flux coherent atomic sources.  Although pulsed coherent atomic sources are available in the form of Bose-Einstein condensates, continuous coherent atomic sources are at present not available.  This thesis investigates two methods of replenishing the Bose-Einstein condensate to enable a continuous, coherent atomic source to be extracted from the condensate: a pumped atom laser.

We theoretically study two possible mechanisms for replenishing the Bose-Einstein condensate.  



The first drives the process optically, and has the advantage that losses from the system are not a fundamental element of the process.  Blah blah blah


We show that it is possible to 



Fundamentally, there are two different 